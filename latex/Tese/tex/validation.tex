\chapter{Validation}

\section{Quickcheck}

The current FreeST compiler features an algorithm for checking the bisimilarity
of simple grammars, which we use for testing. The testing process takes a suite
of randomly generated types---a small subset of FreeST's types, based on the
syntax presented in \cref{fig:syntax-types}---leveraging the Quickcheck
library~\cite{DBLP:conf/icfp/ClaessenH00} to ensure these types have specific
properties. Formal proofs regarding decidability of type formation
and equivalence can be found elsewhere~\cite{DBLP:conf/esop/PocasCMV23}.
\section{Properties and testing}

An arbitrary type generator is defined using the \lstinline{Arbitrary} typeclass, employing the \lstinline{frequency} function to generate type operators with specific probabilities. Variables are selected from a predefined range, abstractions are created by generating a variable, a kind, and a sub-type, and applications are formed by recursively generating two sub-types. The \lstinline{sized} function is used to control the size of the generated types, ensuring manageable recursion depth. For better statistics we ensure proper distribution of type constructors. The list of properties can be found in \cref{tab:properties}. A total of 200.000 tests were made for each property.

Data was collected on a machine equipped with an Apple M3 Pro and 18GB of RAM, and tested with Haskell's version 9.6.3.

While randomly generated types facilitate a robust analysis, certain properties,
such as the type-formation preservation property and bisimilarity of simple
grammars, prove challenging to test comprehensively. The difficulty arises from
the simplicity of our generator and the inherent low probability that randomly
generated test cases yield types that are both well-formed and reduce.
Therefore, most of the tests cases do not satisfy the precondition
$\istype{} T \kind$ and $\betared{T}{U}$, and Quickcheck ends up discarding
2.000.000 tests for the last two properties. To achieve better results, more
complex generators, tailored to specific properties, would be required. Such
generators are often challenging to design and implement.


\LIMPA