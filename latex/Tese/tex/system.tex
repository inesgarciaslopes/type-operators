\chapter{System $\FMuSemiOmega$}

\todo{What will we talk about in this chapter?}

\section{From System F to $\FMuSemi$}\label{sec:system}

Let $\alpha$,$\beta$ range over a countable set of type variables. The types of System F are given by the grammar:
\begin{align*}
  \typec T \grmeq & \typec{\alpha}
  \grmor \function TT
  \grmor \foralltinfix\alpha\tkind T 
  && (\FMu)
\end{align*}

A type of the form $\function TT$ is called a function type, and a type of the form $\foralltinfix\alpha\tkind T $ is called a universal type. In $\foralltinfix\alpha\tkind T $, the type variable $\alpha$ is bound with kind $\tkind$, the functional base kind of this system. Being interested in describing communication protocols, system F is rather simple and limited. Adding recursion lets us at least describe types such as $\tmuinfix\alpha\tkind\Int$. However, this expresses no real pratical meaning for communication protocols. Proving system $\FMu$ is not expressive enough for our proposes.

Building on top of this, with recursion and session types, the types of system $\FMuDot$ are given by the following grammar:
\begin{align*}
  \typec T \grmeq & (\FMu)
  \grmor \tmuinfix{\alpha}{\kind}{T} 
  \grmor \MSG TT
  \grmor \tchoice{l_i:T_i}
  \grmor \End 
  && (\FMuDot)
\end{align*}

Now we can express recursion through $\tmuinfix{\alpha}{\kind}{T}$, send or receive messages through the message construtors $\MSG TT$ and offer choices through $\tchoice{l_i:T_i}$, where $l_i$ are labels. The type $\End$ terminates a communication. We also need to introduce the kind of session types, kind $\skind$, to control type formation when dealing with both session and functional types alltogether. Now we can write a channel endpoint that receives a stream of integer values, $\tmuinfix{\alpha}{\skind}{\extchoice\records{\donel\colon\End,\morel\colon?\Int;\alpha}}$: if the label $\donel$ is selected, the channel closes, whereas if the label $\morel$ is chosen, we receive an integer value followed by the rest of the stream.

\begin{align*}
  \typec T \grmeq & (\FMu)
  \grmor{} \MSGn T
  \grmor{} \tchoice{l_i:T_i}
  \grmor \End
  \grmor{} \semit TT 
  \grmor \Skip
  && (\FMuSemi)
\end{align*}

%
% Example of higher-order channel
% instead of streaming trees of integers, it
% streams trees \emph{of channels} that send a boolean value:
% $\tmuinfix{\alpha}{\skind}{\&\{\leafl: \Skip, \nodel:
%   \alpha;?(\OUT\Bool\End);\alpha\}}\typec;\End$.
%

%A goal of this paper is the integration of higher-order context-free session types into system $\FMuOmega$. We want to be able to abstract the type that is received on a tree channel, which is now possible by writing $\tabs \alpha \tkind{\tmuinfix{\beta}{\skind}{\&\{\leafl: \Skip, \nodel: \beta;?\alpha;\beta\}}}\typec;\End$, where $\tkind$ is the kind of functional types.

The next step of construction leads us to context-free session types, embodied in the type system $\FMuSemi$. We'll be needing a new construct for sequential composition $\semit TU$, and a new type $\Skip$, acting as the neutral element of sequential composition.%~\cite{DBLP:conf/icfp/ThiemannV16}. 
%The message constructors are now unary ($\INn T$ and $\OUTn T$) rather than binary.
Now we can write a type $\tmuinfix{\alpha}{\skind}{\extchoice\rchannel{\leafl}{\Skip}{\nodel}{\alpha\Semi\ \INp\Int\Semi\ \alpha}}\Semi\End_{\typec{?}}$ which describes a protocol for safely streaming integer trees on a channel. The channel presents an external choice $\extchoice$ with two labels: if the continuation of the channel is $\leafl$, then no communication occurs but the channel is still open for further composition whereas, if the continuation of the channel is $\nodel$, then we have a left subtree, followed by an integer and a right subtree. When the whole tree is received, the channel is closed. It is also important to distinguish type $\End_{\typec{?}}$ from $\Skip$---the former represents the closure of a channel, where no further communication is possible, while the latter allows continuing the communication. 

We've reached a point in time where we want to move beyond context-free session types, namely, we are interested in abstracting the type that is received on the tree channel, by writing $\tabs{\beta}{\tkind}{\tmuinfix{\alpha}{\skind}{\extchoice\rchannel{\leafl}{\Skip}{\nodel}{\alpha\Semi\ \INp\beta\Semi\ \alpha}}\Semi\Wait}$. This desire brings the necessity for type abstractions, $\tabs\alpha\kind T$, and kinds of the form $\karrow \kind \kind$. This invites higher-order kinds into play which in turn leads to the introduction of type operators into our language.

%\begin{equation*}
%\streamt = \tabs\alpha\tkind{(\tmuinfix{\beta}{\skind}{\extchoice\records{\donel\colon\End,\morel\colon\IN\alpha\beta}})}
%\end{equation*}
%where $\typec \alpha$ can be instantiated with the desired type; 
%in particular, $\tapp \streamt \Int$ would be equivalent to the aforementioned $\intstreamt$.

\section{Adding Type Operators}

%fix later.

\begin{figure}[t]
	\centering
	%\begin{minipage}{.45\textwidth}
		\begin{align*}
			\TT \grmeq &&&&&&&& \text{\small Type}\\
			& \tiota && \text{\small type constructor} && \tabs \alpha\kind T && \text{\small  type-level abstraction}\\
			& \talpha && \text{\small  type variable} && \tapp TT && \text{\small  type-level application}
        \end{align*}
        \begin{align*}
           \tiota \grmeq &&&&&&&&&&& \text{\small\qquad\quad {Type constructor}}\\
            % Functional
			&\Arrow && \karrow \kast {\karrow {\kast}{\tkind}} &&  \text{\small arrow} && \typec{\varrecs{\overline{l_i}}} && \kindc{\overline{{\kast \Rightarrow}}\,\tkind} && \text{\small record/variant}\\
			&\tmu \kind && \karrow {(\karrow \kind \kind)} {\kind} && \text{\small recursive type}
			&& \Sharp && \karrow \kast \skind &&\text{\small input/output} \\
            % Session types
            &\Semi && \karrow \skind {\karrow \skind \skind} && \text{\small seq.composition}
			 && \typec{\odot{\{\overline{l_i}\}}} && \kindc{\overline{{\skind
						\Rightarrow}}\,\skind} && \text{\small internal/external choice}\\
            &\quant \kind && \karrow {(\karrow \kind \kast)} \tkind && \text{\small exists/forall}
            && \Dual && \karrow{\skind}{\skind} && \text{\small dual operator}\\
            &\End_{\typec{\sharp}} && \skind && \text{\small wait/close} && \Skip && \skind && \text{\small skip} 
			%&\Skip && \skind && \text{\small skip} &&&
        \end{align*}
    \begin{align*}
    \typec{\varrecs{}} \grmeq{} \typec{\records{}} \!\grmor\! \typec{\variants{}}
    &&
    \typec\sharp \grmeq{} \typec? \!\grmor\! \typec!
    &&
    \choice \grmeq{} \intchoice \!\grmor\! \extchoice
    &&
    \quant\kind \grmeq{} \typec{\exists_\kind} \!\grmor\! \typec{\forall_\kind}
  \end{align*}
    %\end{minipage}
    \caption{The syntax of types and constructors.}
    \label{fig:syntax-types}
\end{figure}


%%% Local Variables:
%%% mode: latex
%%% TeX-master: "samplepaper"
%%% End:


The types of system $\FMuSemiOmega$ are an extension of system $\FMuOmega$, accounting for higher-order context-free session types. A type can be either a type constructor $\tiota$, a type variable $\talpha$, an abstraction $\tabs{\talpha}{\kind}{\typec{T}}$ or an application $\tapp{\typec{T}}{\typec{T}}$. The type constructors consist on a function operator $\Arrow$, a record or variant $\typec{\varrecs{\overline{l_i}}}$, recursive operator $\tmu \kind$, message operator $\Sharp$, sequential composition operator $\Semi$, choice operator $\typec{\odot{\{\overline{l_i}\}}}$, quantifier operators $\quant \kind$, duality operator $\Dual$, waiting or closing operators $\End_{\typec{\sharp}}$ and the no action operator $\Skip$. Their respective kinds can be found in \cref*{fig:syntax-types}. 

In our system, a \emph{kind} $\kind$ is either a base kind $\kast$---which is either a session or a functional kind, $\skind$ or $\tkind$, respectively---or a higher-order kind  $\karrow \kind \kindp$. A \emph{proper type} refers to a type that has a base kind. In contrast, \emph{type operators} act upon types to create more complex types, and are associated with higher-order kinds. For example, the type operator $\Arrow$ takes two types, $\typec{T}$ and $\typec{U}$, and constructs a new type, a function type $\Arrow \typec{T} \typec{U}$, which we sometimes write in infix notation as $\typec{T}\ \Arrow\ \typec{U}$. 

\todo{[Maybe not here. Section with differences with freest?] Our goal is to expand the programming language FreeST, currently limited to types of kind $\kast$, to higher-order kinds.}

%and we identify types up to renaming of bound variables; thus, ∀α.α → α and ∀β.β → β are the same type. 

\todo{It is to note that the kind of the operator $\quant \kind$ has been promoted to attend for polymorphic session types.[Confusing here? Maybe in a section where show the differences with freest?]}

Observe that $\tmuinfix{\alpha}{\kind}{T}$ is syntactic sugar for $\typec{\tmu\kind (\tlambda{\alpha}{\kind}{T})}$ and similarly for $\tquantinfix\alpha\kind \TT$.

Not all types are of interest---for example $\tapp{\Dual}{\forallt{\skind}{\tabs{\alpha}{\skind}{\alpha}}}$ since the dual operator makes no sense being applied to a functional type. We'll tackle this concern later on in section x. 

\todo{paragrafo de ligação.}


\section{Reduction and Normal Form}
% !TeX root = samplepaper.tex
\begin{figure}[t]
	\declrel{Type reduction}{$\lambdared TT$}
	\begin{mathpar}
		\infer[\rseqone]{}{\lambdared{\semit \Skip T}{T}}
		
		\infer[\rseqtwo]{\lambdared{T}{V} \\ \typec{T} \neq \typec{T_1; T_2}}{\lambdared{\semit TU}{\semit VU}}
		
		\infer[\rassoc]{}{\lambdared{\semit {(\semit TU)}V}{\semit T {(\semit UV)}}}
		
		\infer[\rmu]{}{\lambdared{\tapp{\tmu{k}}\TT}{\tapp{T}({\tapp{\tmu{k}} \TT})}}

		\infer[\rbeta]{}{\lambdared{\tapp{(\tabs{\alpha}{\kind}{T})}{U}}{\TT\subs U\alpha}}

        \infer[\rabs]{\lambdared TU}{\lambdared{\tabs\alpha\kind T}{\tabs\alpha\kind U}}
		
		\infer[\rtappl]{\lambdared TU}{\lambdared{TV}{UV}}

		\infer[\rdsemi]{}{\lambdared{\tapp \Dual {(\semit T U)}}{\semit {\tapp \Dual
					T}{\tapp \Dual U}}}
		
		\infer[\rdskip]{}{\lambdared {\tapp \Dual \Skip}{\Skip}}
		
		%\infer[\rdend]{}{\lambdared {\tapp \Dual \End}{\End}}

        \infer[\rdwait]{}{\lambdared{\tapp \Dual \Wait}{\Close}}

        \infer[\rdclose]{}{\lambdared{\tapp \Dual \Close}{\Wait}}
		
		\infer[\rdin]{}{\lambdared{\tapp \Dual {(\tapp ?T)}}{\tapp !T}}
		
		\infer[\rdout]{}{\lambdared{\tapp \Dual {(\tapp !T)}}{\tapp ?T}}
		
		\infer[\rdextc]{}{\lambdared{\tapp {\Dual} {( {\&\{\overline{l_i:T_i}\}})}}{\oplus\{\overline{l_i:\Dual(T_i)}\}}}
		
		\infer[\rdintc]{}{\lambdared{\tapp {\Dual} {( {\oplus\{\overline{l_i:T_i}\}})}}{\&\{\overline{l_i:\Dual(T_i)}\}}}
  
		\infer[\rdctx]{\lambdared TU \\ \typec{T} \neq \typec{T_1;T_2}}{\lambdared{\tapp \Dual T}{\tapp \Dual U}}
		
		\infer[\rddvar]{}{\lambdared{\tapp {\Dual} {(\tapp \Dual (\alpha\ \TT_1\ldots \TT_m))}}{\alpha\ \TT_1\ldots \TT_m}}

        \infer[\rdexists]{}{\lambdared{\tapp {\Dual}{(\texists{\kind} \TT)}}{}}{\tforall{\kind} \TT}

        \infer[\rdforall]{}{\lambdared{\tapp {\Dual}{(\texists{\kind} \TT)}}{}}{\tforall{\kind} \TT}
	\end{mathpar}
	\caption{Type reduction.}
	\label{fig:reduction}
\end{figure}

%%% Local Variables:
%%% mode: latex
%%% TeX-master: "samplepaper"
%%% End:

For now, let's define the \emph{weak head normal form} of a type; We do this so by defining a system of reduction rules in \cref*{fig:reduction}. This definition provides reduction steps for sequential composition, recursive types (one-step unfold) and $\Dual$ types. It is important to note that there are no reductions under lambda abstractions: thus, when invoking rule $\rbeta$, only a \textit{substitution} is performed. In lambda calculus [ref], substitution replaces all free occurrences of a variable in a expression with another expression, written as $[\typec{x} \rightarrow \typec{E_1}]\typec{E_2}$. A new Dual rule was introduced, $\rddual$ that is $\lambdared{\tapp \Dual (\tapp \Dual \TT)}{\TT}$, providing that $\rdctx$ was equally corrected such that $\TT\neq \tapp \Dual {\typec{T_1}}$. This system is that of Poças et al.[] made confluent by adding the proviso that $\typec{T} \neq \typec{\seqcomp{T_1}{T_2}}$ to rules \rseqtwo and \rdctx. 

\textbf{(Confluence) } If $\afunction{T}{U}$ and $\afunction{T}{V}$, then $\afunctionstar{U}{W}$ and $\afunctionstar{V}{W}$.\\
\emph{Confluence} states that, if there are two distinct reductions for a given type, $\afunction{T}{U}$ and $\afunction{T}{V}$, then both paths will eventually converge into the same final reduced type $\typec{W}$. One may visualize this property in the following latice:

\begin{center}
    \begin{tikzpicture}
    % Define matrix
        \matrix (m) [matrix of math nodes, row sep=3em, column sep=3em]
        {
            & \typec{T} \\
            \typec{U} && \typec{V} \\
            & \typec{W} \\
        };
        % Draw arrows
        \draw[->] (m-1-2) -- (m-2-1) node[midway, above] {};
        \draw[->, dashed] (m-2-1) -- (m-3-2) node[near end, below] {$*$};
        \draw[->] (m-1-2) -- (m-2-3) node[midway, above] {};
        \draw[->, dashed] (m-2-3) -- (m-3-2) node[near end, below] {$*$};
    \end{tikzpicture}
\end{center}
For example, the type $\tappdual{(\seqcomp{\Skip}{T})}$ has two possible reductions: one via rule \srule{r-dctx} and another by applying rule \srule{r-d} first, followed by \srule{r-seq2} and then \srule{r-seq1}.

\begin{align*}
  \tappdual{(\seqcomp{\Skip}{T})} & \rightarrow \tappdual{T}  & \srule{r-dctx}\\
  \\
  \tappdual{(\seqcomp{\Skip}{T})} & \rightarrow \seqcomp{\tappdual{\Skip}}{\tappdual{T}} & \srule{r-d}\\
  & \rightarrow \seqcomp{\Skip}{\tappdual{T}} & \srule{r-seq2}\\
  & \rightarrow \tappdual{T} & \srule{r-seq1}
\end{align*}

This does not break confluence. But, without the proviso, we found that some types would not preserve the property of confluence. For example, the type $\tappdual{(\seqcomp{(\seqcomp{T}{U})}{V})}$ has two possible reductions: one via rule \srule{r-dctx} and then rule \srule{r-d} and other by applying rule \srule{r-d} first, followed by \srule{r-seq2} and then \srule{r-assoc}.

\begin{align*}
  \tappdual{(\seqcomp{(\seqcomp{T}{U})}{V})} & \rightarrow \tappdual{(\seqcomp{T}{(\seqcomp{U}{V})})} & \srule{r-dctx}\\
  & \rightarrow \seqcomp{\tappdual{T}}{\tappdual{(\seqcomp{U}{V}})} & \srule{r-d}\\
  \\
  \tappdual{(\seqcomp{(\seqcomp{T}{U})}{V})} & \rightarrow \seqcomp{\tappdual{(\seqcomp{T}{U})}}{\tappdual{V}} & \srule{r-d}\\
  & \rightarrow \seqcomp{(\seqcomp{\tappdual{T}}{\tappdual{U}})}{\tappdual{V}} & \srule{r-seq2}\\
  & \rightarrow \seqcomp{\tappdual{T}}{(\seqcomp{\tappdual{U}}{\tappdual{V}})} & \srule{r-assoc}
\end{align*}
In this case, the two reductions do not converge into the same type since neither will $\tappdual{(\seqcomp{U}{V})}$ or $\seqcomp{\tappdual{U}}{\tappdual{V}}$ reduce any further. We found this true for types whose head is a sequential composition. Then, we decided to add the predicate $\typec{T} \neq \typec{\seqcomp{T_1}{T_2}}$ to rules \srule{r-seq2} and \srule{r-dctx} to express this limitation. Arriving at the system expressed by \cref*{fig:reduction}, our variant features a single reduction path, thus confluence immediately follows.

\begin{align*}
  \tappdual{(\seqcomp{(\seqcomp{T}{U})}{V})} & \rightarrow \seqcomp{\tappdual{(\seqcomp{T}{U})}}{\tappdual{V}} & \srule{r-d}\\
  & \rightarrow \seqcomp{(\seqcomp{\tappdual{T}}{\tappdual{U}})}{\tappdual{V}} & \srule{r-seq2}\\
  & \rightarrow \seqcomp{\tappdual{T}}{(\seqcomp{\tappdual{U}}{\tappdual{V}})} & \srule{r-assoc}\\
\end{align*}

%With respect to Po\c{c}as et al. , we promoted polymorphism to kind $\skind$, which entails the introduction of the existential type $\texists \kind$, dual of $\tforall \kind$, and the corresponding reduction rules.

% !TeX root = samplepaper.tex
\begin{figure}
	\declrel{Weak head normal form }{$\iswhnf T$}
	\begin{mathpar}
  \inferrule[w-const0]{}{\iswhnf{\tiota}}
  \and
  \inferrule*[lab=w-const1]{\tiota \neq \Semi\ , \tmu{\kind}\ ,\Dual \\ m \geq 1  }{\iswhnf{\tapp{\tiota}{T_1...T_m}}}
  \and
  \inferrule[w-seq1]{}{\iswhnf{\tapp{\Semi}{T}}}
  \and
  \inferrule*[lab=w-seq2]{\iswhnf{T_1} \\ \typec{T_1} \neq \seqcomp{U}{V}\ , \Skip \\ m \geq 2 }{\iswhnf{\tapp{\Semi}{T_1...T_m}}}
  \and
  \inferrule*[lab=w-var]{\mathtt\mathit m \geq 0  }{\iswhnf{\tapp{\talpha}{T_1...T_m}}}
  \and 
  \inferrule[w-abs]{}{\iswhnf{\tabs{\talpha}{\kind}{\typec{T}}}}
  \and
  \inferrule*[lab=w-dual]{\iswhnf{T_1} \\ \mathtt\mathit m \geq 0 \\ \typec{T_1} \neq \quant\kind, \Skip, \Wait, \Close, \Sharp\typec{U},\seqcomp{U}{V}, \typec{\odot}{\trecord{l_i}{U_i}}, \tappdual{(\tapp{\talpha}{U_1...U_m})}}{\iswhnf{\tapp{\Dual}{T_1...T_m}}}
	\end{mathpar}
	\caption{Weak head normal form.}
    \label{fig:whnf}
\end{figure}
%%% Local Variables:
%%% mode: latex
%%% TeX-master: "samplepaper"
%%% End:


Another way to express weak head normal form is through inference-based characterisation. Any type operator $\typec{\iota}$ or abstraction $\tabs{\alpha}{\kind}{\TT}$ is in weak head normal form. Note that, we consider the partially-applied sequential composition $\tapp{\Semi}{T}$ to be in normal form, as expressed by rule $\wseqone$ in \cref*{fig:whnf}. The notion of normal form comes from the lambda calculus. This lets us know when an expression is fully evaluated. We say a type $\TT$ is in \emph{normal form}, denoted $\iswhnf{T}$, if it has been completely reduced, \textit{i.e.}, no further reductions are possible. In other words, both characterisations (\cref*{fig:reduction} and \cref*{fig:whnf}) are equivalent; $\iswhnf{T}$ is equivalent to irreducibility, $\typec{T} \nrightarrow$. 

Then we can say that type $\typec{T}$ \emph{normalises} to type $\typec{U}$, written $\isnorm{T}{U}$, if $\iswhnf{U}$ and $\typec{U}$ is reached from $\typec{T}$ in a finite number of reduction steps. The predicate $\isnormalised{T}$ means that $\isnorm{T}{U}$ for some $\typec{U}$.
Note that not all types normalise, \ie some have an infinite sequence of reductions, such as $\TT = \typec{(\lambda x. xx)(\lambda x. xx)} \longrightarrow \typec{(\lambda x. xx)\subs{(\lambda x. xx)} {x}} = \TT \longrightarrow \typec{\cdots}$---which is stuck in a loop of reductions to itself via the application of rule \rbeta---and $\UT = \typec{\tmu\skind (\tlambda{\alpha}{\skind}{\alpha;\Skip})} \rightarrow (\tlambda{\alpha}{\skind}{\alpha;\Skip})\UT \rightarrow \typec{U;\Skip} \rightarrow_2 \typec{U;\Skip;\Skip} \rightarrow \typec{\cdots}$---which successively applies rules \rmu and \rbeta in combination with \rseqtwo, resulting in the unending addition of a trailing $\Skip$.

\section{Type Formation}

Finally, we introduce \emph{well-formed types}. We use $\judgementrelctx{\Delta}{\typec{T}}{\colon}{\kind}$ to denote that $\typec{T}$ is a \emph{well-formed type} with kind $\kind$ under the kinding context $\Delta$, a map from type variables to kinds. A type variable $\talpha$ has kind $\kind$ if $\tbind\alpha\kind\in{\Delta}$. An abstraction $\tabs\alpha \kind \TT$ has kind $\karrow \kind {\kind'}$ if $\TT$ is well-formed. Note that $\Delta + \tbind\talpha\kind$ in rule \ktabs represents updating the kind of the type variable $\talpha$ to a new kind $\kind$ in the context $\Delta$ if $\tbind\talpha\kindp\in\Delta$ for some $\kindp$, or storing the kind of $\talpha$ in the context $\Delta$ if otherwise. Finally, rule \ktapp states that an application $\tapp T U$ is well-formed if $\TT$ and $\UT$ are types and $\tapp TU$ normalises, that is, $\isnormalised {\tapp TU}$.

% !TeX root = main.tex

\begin{figure}[t!]
  \declrel{Type formation}{$\istype \Delta T \kind$}
  \begin{mathpar}
    \infer[\kconst]{}{\istype \Delta \iota {\kind_\iota}}
    \quad
    \infer[\kvar]{\tbind\alpha\kind\in{\Delta}}
    {\istype \Delta \alpha \kind}
    \quad
    \infer[\ktabs]{\istype {\Delta+\tbind\alpha \kind} {T} {\kind'}}
    {\istype \Delta {\tabs\alpha \kind \TT} {\karrow \kind {\kind'}}}
    \quad
    \inferrule[\ktapp]{\istype \Delta {T} {\karrow \kind {\kind'}} \;\;\, \istype \Delta U \kind \;\;\, \isnormalised {\tapp TU}}
    {\istype \Delta {\tapp T U} {\kind'}}
  \end{mathpar}
  \caption{Type formation.}
  \label{fig:type-formation}
\end{figure}

%%% Local Variables:
%%% mode: latex
%%% TeX-master: "main"
%%% End:


\subsection{Decidability of type formation}
%We prove decidability of type formation, imposing a restriction to kind * for recursive types, also adopted by Poças et al.

The rule \ktapp for type formation in \cref{fig:type-formation} involve determining if an application type $\TT\UT$ normalises. However, we just saw that some types do not normalise and we do not want an algorithm that does not terminate. 
Poças et al.~\cite{DBLP:conf/esop/PocasCMV23} proposes a two-step solution to this problem.
The first stage is the introduction of the concept of \emph{pre-kinding}. We denote this as $\prekind \Delta T \kind$, that is, $\typec{T}$ is pre-kinded with kind $\kind$ under the kinding context $\Delta$. The rules for pre-kinding are in \cref{fig:pre-kinding}. They differ from the rules for type formation in that, in rule \pktapp, there is no verification of the normalisation of $\TT\UT$. Pre-kinding excludes some (but not all) types that do not normalise, as is the case of $\typec{(\tlambda{\alpha}{\kind}{\alpha \alpha})(\tlambda{\alpha}{\kind}{\alpha \alpha})}$ in \cref{fig:ex-pre-kinding}.

% !TeX root = samplepaper.tex

\begin{figure}[t]
  \declrel{Pre-kinding}{$\prekind \Delta T \kind$}
  \begin{mathpar}
    \infer[\pkconst]{}
    {\prekind \Delta \iota {\kind_\iota}}
    \quad
    \infer[\pkvar]{\tbind\alpha\kind\in{\Delta}}{\prekind \Delta \alpha \kind}
    \quad
    \infer[\pktabs]{\prekind {\Delta+\tbind\alpha \kind} {T} {\kind'}}{\prekind \Delta {\tabs\alpha \kind \TT} {\karrow \kind {\kind'}}}
    \quad
    \inferrule[\pktapp]{\prekind \Delta {T} {\karrow \kind {\kind'}} \;\;\, \prekind \Delta U \kind}{\prekind \Delta {\tapp T U} {\kind'}}
  \end{mathpar}
  \caption{Pre-kinding.}
  \label{fig:pre-kinding}
\end{figure}

%%% Local Variables:
%%% mode: latex
%%% TeX-master: "samplepaper"
%%% End:


\begin{figure}[t]
	\begin{equation*}
		\inferrule*[rightstyle = {\scriptsize \sc}, right = PK-TApp]{
			\inferrule*[rightstyle = {\scriptsize \sc}, right = PK-TAbs]{
				\inferrule*[rightstyle = {\scriptsize \sc}, right = PK-TApp, rightskip=9em]{
					\infer[\textsc{\scriptsize PK-Var}]{
						\kind = \karrow {\kind''} {\kind'}
					}{
						\prekind {\typec{\alpha}:\kind} {\alpha} {\karrow {\kind''} {\kind'}}
					}
					\quad
					\infer[\textsc{\scriptsize PK-Var}]{
						\bot\quad (\kind \neq \kindc{\kind''})}{
						\prekind {\typec{\alpha}:\kind} {\alpha} {\kind''}
					}
				}{
					\prekind {\typec{\alpha}:\kind} {\alpha \alpha} {\kind'}
				}
			}
			{
				\prekind {} {\tlambda{\alpha}{\kind}{\alpha \alpha}} {\karrow {\kind} {\kind'}}
			}
			\qquad
			\inferrule*[]{
				\vdots}{
				\prekind {} {\tlambda{\alpha}{\kind}{\alpha \alpha}} {\kind}
			}
		}
		{
			\prekind {} {(\tlambda{\alpha}{\kind}{\alpha \alpha})(\tlambda{\alpha}{\kind}{\alpha \alpha})} {\kind'}
		}
	\end{equation*}
	\caption{Example of an unsuccessful derivation for a pre-kind goal.}
    % $\prekind {} {(\tlambda{\alpha}{\kind}{\alpha \alpha})(\tlambda{\alpha}{\kind}{\alpha \alpha})} {\kind'}$.}
	\label{fig:ex-pre-kinding}
\end{figure}

For a type which is pre-kinded, termination of $\isnormalised {\TT}$ is guaranteed. Some recursive types are problematic for normalisation, as the application of reduction might not decrease their size. For example, the type $\tmu\skind\ (\tabs\alpha\skind {\alpha;\Skip})$ is pre-kinded but successive reduction steps---via \rmu and \rbeta---keep adding $\typec{;}\Skip$ to the tail of the type so we must conclude that it does not normalise.
When dealing with normalisation, we separate the treatment of recursive types from the remaining types.
In particular, we divide the reduction rules in two groups: $\muarrow$ refers to reductions that use the $\rmu$ rule and $\lseqarrow$ refers to reductions that never invoke the $\rmu$ rule. Thus, $\betaarrow\ =\ \lseqarrow\cup\muarrow$. We may now lift this notion to normalisation, denoted by $\lseqnormalred TU$ and $\munormalred TU$ respectively.

In order to check if a type $\TT$ is well-formed, we first determine if $\prekind{} T \kind$ for some $\kind$. If $\TT$ fails to be pre-kinded, it is not kinded either. Otherwise, we check whether $\istype{} T \kind$, which involves determining if the application types within $\TT$ normalise. 

\todo{why was it necessary to check all application types ? example.}

\subsection{Restrictions to Recursion}
The approach to determine if a type normalises seeks infinite reduction sequences. In the case of recursive types, such sequences would have a finite number of $\beta$-reductions between two $\mu$-reductions.
$\TT = \lseqnormalred{\typec{T_0}}{\typec{T'_0}}\mured{}{\typec{T_1}}\lseqnormalred{}{\typec{T'_1}}\mured{}{\typec{T_2}}\lseqnormalred{}{\typec{T'_2}}\mured{}{\cdots}$.
If $\typec{T'_i}$ does not reduce by any $\mu$-reduction, we can conclude that
$\typec{T}$ normalises. Otherwise, since $\tmu\kast \UT$ is restricted to a base
kind $\kast$, it must reduce by one of following cases.
\begin{align*}
\typec{T'_i} &= \mured{\tapp{\tmu\kast}\UT}{\tapp\UT{(\tapp{\tmu\kast}\UT)}} & (\rmu)
\\
\typec{T'_i} &= \mured{\semit{(\tapp{\tmu\kast}\UT)}{V}}{\semit{(\tapp U{(\tapp{\tmu\kast}\UT)})}{V}} & (\rseqtwo)
\\
\typec{T'_i} &= \mured{\tdual{(\tapp{\tmu\kast}\UT)}}{\tdual{(\tapp U {(\tapp{\tmu\kast}\UT)})}} & (\rdctx)
\\
\typec{T'_i} &= \mured{\semit{(\tdual{(\tapp{\tmu\kast}\UT)})}{V}}{\semit{(\tdual{(\tapp U {(\tapp{\tmu\kast}\UT)})})}{V}} & (\rseqtwo + \rdctx)
\end{align*}
We can easily notice that expression $\tmu\kast\UT$ reappears after the $\mu$-reduction, indicating potential infinite sequences. We can detect these by tracking occurrences of $\tmu\kast\UT$ and halting if a repetition is found.
%The following pseudo-code illustrates the process for determining if $\typec{T}$ normalises: 
%
%\begin{lstlisting}
%    normalises(visited, t) = 
%        if reducesByBSD(t) then
%            normalises(visited, t') -- t -> t'
%        else if memberOf(t, visited) then
%            Nothing -- found an infinite sequence
%        else if reducesByMu(t) then
%            normalises(visited', t') -- update visited set with t
%        else t
%\end{lstlisting}
%%% Local Variables:
%%% mode: latex
%%% TeX-master: "42-CR"
%%% End:



\section{Type equivalence + Decidability of type equivalence}

\todo{paragrafo de ligação}

Type equivalence allows us to check whether two types, even if syntactically different, correspond to the same protocol. 
It is expected that two types that are alpha-congruent are equivalent, like for example $\tlambda{\alpha}{\kind}{\alpha}$ and $\tlambda{\beta}{\kind}{\beta}$. However, the task of checking if two types are equivalent may involve substitutions on-the-fly as one crosses along the types. We will avoid this by performing a renaming operation once on both types, right at the beginning of the type equivalence checking process. 

Inspired by the renaming approach of Gauthier and Pottier \cite{GauthierP04}, we introduce \textit{minimal renaming}, which uses the least amount of variable names necessary.
%This notion of minimal renaming enables the translation of types into simple grammars.
This renaming operation, defined in figure \ref{fig:rename} consists on replacing a type \typec{T} by its minimal alpha-conversion. We assume that a countable ordered set of (fresh) type variables, $\{\typec{\upsilon_1},\ldots,\typec{\upsilon_n},\ldots\}$, is available. We rename bound variables in \typec{T} by the smallest possible variable available, \textit{i.e.}, the first which is not free on the type. Note that the $S$ parameter in the definition corresponds to the set of unavailable type variables for renaming.  Also, by $\first(S)$ we mean the smallest variable not in the set $S$. The notion of variable substitution, $\typec{T}\subs{v}{\talpha}$, is the one from literature \cite{tapl}. 
and free variables of a type, $\fv(\TT)$, are the standard ones from literature \cite{tapl}.

\begin{figure}[h]
  \declrel{Type renaming}{$\rename_S(\TT)$}
  \begin{align*}
    \rename_S(\typec\iota)&=\typec\iota
    \\
    \rename_S(\typec\alpha)&=\typec\alpha
    \\
    \rename_S(\tabs{\alpha}{\kind}{T}) &=
    \tabs{\upsilon}{\kind}{{\blk{\rename_S(}}T\subs\upsilon\alpha})
    \quad\text{where } \typec{\upsilon} = \first(S\cup\fv(\tabs{\talpha}{\kind}{T}))
    \\
    \rename_S(\tapp{T}{U}) &= \rename_{S\cup\fv(\UT)}(\TT)\rename_S(\UT)
  \end{align*}
  \caption{Type renaming.}
  \label{fig:rename}
\end{figure}

%Po\c{c}as et al propose three possible ways to check whether two (renamed) types are equivalent:
%\begin{itemize}
%    \item by using a set of coinductive rules;
%    \item by means of a bisimulation on types, built from a labelled transition system;
%    \item by translating types to grammars and checking bisimulation of grammars.
%\end{itemize}
%
%We focus on the third approach, which we detail in.

%% !TeX root = main.tex

\begin{figure}[h]
    \declrel{Type absorbing}{$absorbing(T)$}
    \begin{mathpar}
      \infer[\kconst]{}{\End \Sharp}
    \end{mathpar}
    \caption{Type absorbing.}
    \label{fig:rename-absorbing}
  \end{figure}
  
  %%% Local Variables:
  %%% mode: latex
  %%% TeX-master: "main"
  %%% End:
  

Following Poças et al.~\cite{DBLP:conf/esop/PocasCMV23}, the problem of checking whether two (renamed) types are equivalent is reduced to translating types into grammars and checking bisimilarity. A grammar in \emph{Greibach normal form} \cite{AutebertG84} is a tuple $(\mathcal{T, N}, \ntgamma, \mathcal{R})$, where:
\begin{itemize}
	\item $\mathcal{T}$ is a finite set of terminal symbols, $\tsymc{a}$, $\tsymc{b}$, $\tsymc{c}$; 
	\item $\mathcal{N}$ is a finite set of non-terminal symbols, $\Xnt, \Ynt, \Znt$;
	\item $\ntgamma\in\mathcal{N}^*$ is the starting word;
	\item $\mathcal{R} \subseteq \N \times \T \times \N^*$ is a finite set of production rules.
\end{itemize}

A production rule in $\mathcal{R}$ is written as $\ltsred{\Xnt}{a}{\ntdelta}$. Grammars in GNF are \emph{simple} when, for every non-terminal symbol $\Xnt$ and every terminal symbol $\tsymc{a}$, there is at most one production rule $\ltsred{\Xnt}{a}{\ntdelta}$ \cite{KorenjakH66}.

% !TeX root = samplepaper.tex
\begin{figure}[t]
    \declrel{}{$\word(\typec{T})$}
    \begin{align*}
        \word (\TT) = \begin{cases}
                        \word'(\TT) & \iswhnf \TT\\
                        \emptyword & \normalred T\Skip\\
                        \Ynt, \R := \R \cup \{\gltsred Y a {\gamma\delta} \ |\ \tsymc a, \nsymc \gamma : \gltsred Z a {\gamma}\} & \normalred TU \neq \Skip, \word(\UT) = \Znt \ntdelta
                    \end{cases}
    \end{align*}
  \begin{align*}
    \word'(\talpha\ \typec{T_1}\typec{\ldots}\typec{T_m}) &= \Ynt, \R := \R \cup \{\gltsred Y{\alpha_0}{\emptyword}, \gltsred Y{\alpha_j}{\wordb{\typec{T_j}}\bot}\} 
    \\
    \word'(\Skip) &= \emptyword
    \\
    \word'(\End_{\typec{\sharp}}) &=  \Ynt, \R := \R \cup \{\gltsred Y{\tsymc{\Endl}}{\bot}\}
%    \\
%    \word'(\Close) &= \Ynt, \R := \R \cup \{\gltsred Y\Closel{\bot}\}
    \\
    \word'(\tlambda{\alpha}\kind T) &= \Ynt, \R := \R \cup \{\gltsred Y {\lambda\alpha\colon\kind}{\wordb{\TT}}\}
    \\
    \word'(\tiota) &= \Ynt, \R := \R \cup \{\gltsred Y\iota\emptyword\} \quad
                     \text{where } \typec{\iota}\neq \Skip,\End_{\typec{\sharp}}
    \\
    \word'(\typec{\iota\ T_1\cdots T_m}) &= \Ynt, \R := \R \cup \{\gltsred Y{\iota_j}{\wordb{\typec{T_j}}}\} \quad \text{where }\typec{\iota} = \function{}{}, \typec{\choice\{\overline{l_i}\}}, \typec{\varrecs{\overline{l_i}}}
    \\
    \word'(\MSGn\TT) &= \Ynt, \R := \R \cup \{\gltsred Y{\tsymc{\sharp_1}}{\wordb{\TT}\bot}, \gltsred Y{\sharp_2}{\emptyword}\}
    \\
    \word'(\quant\kind\TT) &= \Ynt, \R := \R \cup \{\gltsred Y{\tsymc{\lquant\kind}}{\wordb{\TT}\bot}
%    , \gltsred Y{\lquant\kind_2}{\emptyword}
    \}
    \\
    \word'(\Semi\ \TT) &= \Ynt, \R := \R \cup \{\gltsred Y{;_1}{\wordb{\TT}}\}
    \\
    \word'(\semit \TT \UT) &= \word(\TT)\word(\UT)
    \\
    \word'(\tdual{(\talpha\ \typec{T_1}\typec{\ldots}\typec{T_m})}) &= \Ynt, \R := \R \cup \{\gltsred Y{\Duall_1}{\wordb{\talpha\ \typec{T_1}\typec{\ldots}\typec{T_m}}}, \gltsred Y{\Duall_2}\emptyword\}
  \end{align*}
  $\Ynt$ is a fresh non-terminal symbol in all cases,\\ 
  $\nsymc \epsilon$ is the empty word,\\
  $\nsymc \bot$ is a non-terminal symbol without productions.\\
  \caption{Function word($\TT$).}
  \label{fig:word}
\end{figure}


%%% Local Variables:
%%% mode: latex
%%% TeX-master: "42-CR"
%%% End:


Our next step is to explain how to convert a type into a simple grammar.

\textbf{Step 1. Construct a simple grammar} Given a type $\TT$ and an initial state, we must invoke function $\word({\typec{T}})$. This function translates types to words of non-terminal symbols. The formal definition of function $\word({\typec{T}})$ can be found in \cref*{fig:word}. If a type $\TT$ is in weak head normal form, the construction of $\word(\TT)$ updates the set of productions of $\TT$, according to one of the cases found in $\word'$. If $\TT$ is not in weak head normal form and normalises to $\Skip$, $\word(\TT)$ returns the empty word; otherwise, if there exists a type $\UT\neq \Skip$ such that $\TT$ normalises to $\UT$, $\word(\UT) = \Znt \ntdelta$ and $\Ynt$ a fresh new terminal, then for each production of $\Znt$ of the form $\gltsred Z a {\gamma}$, $\Ynt$ has a production of the form $\gltsred Y a {\gamma\delta}$. The application of the $\word$ function to a type $\TT$ terminates producing a simple grammar. This is only possible because our well-formed types normalise, and all of its subterms normalise as well. Furthermore, we keep track of already visited types which enable reusing non-terminal symbols, which is crucial for dealing with recursive types.

\textbf{Step 2. Check bisimilarity of types} We check whether two types are equivalent by translating the (renamed) types to a simple grammar, and then checking their bisimilarity, \ie if $\word({\typec{T}}) \gequiv \word({\typec{U}})$. The algorithm used to check bisimilarity of simple grammars is in \cite{AlmeidaMV20}.

Consider the type $
\typec{T_0} = \tabs{\beta}{\tkind}{\tmuinfix{\alpha}{\skind}{\extchoice\rchannel{\leafl}{\Skip}{\nodel}{\alpha\Semi\ \INp\beta\Semi\ \alpha}}\Semi\Wait}$ described in \cref{sec:intro}. We will demonstrate how the construction of $\word(\typec{T_0})$ terminates generating a simple grammar.
Since $\typec{T_0}$ is in weak head normal form, $\word(\typec{T_0})$ returns a fresh symbol, which we call $\nsymc{X_0}$. We also add to the set of productions the production $\gltsred{X_0}{\lambda\beta\colon\tkind}{\wordb{\typec{T_1}}}$, where
$\typec{T_1}$ is the type $\tmuinfix{\alpha}\skind{\extchoice\records{\leafl\colon\Skip,\nodel\colon\semit{{\alpha}}{\semit{\INn{\beta}}{{\alpha}}}}\Semi\Wait}$.

Now $\typec{T_1}$ is not in weak head normal form, so we must normalise it in order to obtain $\typec{T_2}$ such that $\normalred {T_1} {T_2}$. Then, $\word(\typec{T_1})$ returns a fresh non-terminal which we call $\nsymc{X_1}$. To obtain the productions of $\typec{T_1}$, we need to compute $\word(\typec{T_2})$, that returns a fresh symbol $\nsymc{X_2}$. Since $\typec{T_2} = \typec{\extchoice\records{\leafl\colon\Skip,\nodel\colon\semit{T_1}{\semit{\INn {\beta}}{T_1}}}\Semi\Wait}$ we need to compute $\wordb{\typec{T_2}} = \word({\typec{T_3}})\word({\Wait})$. We have that $\wordb{\Wait} = \nsymc{X_4}$ and $\gltsred{X_4}{\tsymc{\waitl}}{\bot}$ but we still need to compute $\wordb{\typec{T_3}}$. This computation results in a fresh non-terminal $\nsymc{X_3}$ with productions $\gltsred{X_3}{\&_1}{\wordb{\Skip}}$ and $\gltsred{X_3}{\&_2}{\wordb{\semit{T_1}{\semit{\INn{\beta}}{T_1}}}}$. Therefore, the transitions for $\nsymc{X_2}$ are $\gltsred{X_2}{\&_1}{\nsymc{X_4}}$ and $\gltsred{X_2}{\&_2}{\nsymc{X_3}\nsymc{X_4}}$.

At last, we must compute $\word(\semit{T_1}{\semit{\INn{\beta}}{T_1}})$, which is a fresh symbol $\nsymc{X_5}$, because this type is not in weak head normal form. This type normalises to $\semit{T_2}{\semit{\INn{\beta}}{T_1}}$, since $\normalred{T_1}{T_2}$, therefore the productions of $\nsymc{X_5}$ are the concatenation of $\word(\typec{T_2})\word(\INn{\beta})\word(\typec{T_1})$. At this point, we know that $\word(\typec{T_2})=\nsymc{X_2}$ and $\word(\typec{T_1})=\nsymc{X_1}$. Thus, we just need to compute $\word(\INn{\beta})$, which is a fresh symbol $\nsymc{X_6}$ with productions $\gltsred{X_6}{?_1}{\wordb{\typec{\beta}}\bot}$ and $\gltsred{X_6}{?_2}{\emptyword}$. Finally, $\word(\typec{\beta})$ is a fresh symbol $\nsymc{X_7}$ with a production $\gltsred{X_7}{{\beta}}{\emptyword}$. This means that $\word(\semit{T_2}{\semit{\INn{\beta}}{T_1}}) = \nsymc{X_2X_6X_1}$, which we write as $\gltsred{X_5}{\&_1}{X_4X_6X_1}$ and $\gltsred{X_5}{\&_2}{X_3X_4X_6X_1}$.

Putting everything together, we obtain the following simple grammar:
%
\begin{align*}
\gltsred{X_0}{\lambda{\beta}\colon\tkind}{X_1}
&&
\gltsred{X_1}{\&_1}{X_4}
&&
\gltsred{X_1}{\&_2}{X_3X_4}
&&
\gltsred{X_2}{\&_1}{X_4}
\\
\gltsred{X_2}{\&_2}{X_3X_4}
&&
\gltsred{X_3}{\&_1}{\emptyword}
&&
\gltsred{X_3}{\&_2}{X_5}
&&
\gltsred{X_4}{\tsymc{\waitl}}{\bot}
\\
\gltsred{X_5}{\&_1}{X_4X_6X_1}
&&
\gltsred{X_5}{\&_2}{X_3X_4X_6X_1}
&&
\gltsred{X_6}{?_1}{X_7\bot}
&&
\gltsred{X_6}{?_2}{\emptyword}
&& 
\gltsred{X_7}{\beta}{\emptyword}
\end{align*}


\todo{this is a problematic example for rename. }
\begin{center}
    \begin{tikzpicture}
    % Define matrix
        \matrix (m) [matrix of math nodes, row sep=3em, column sep=1em]
        {
             \foralltinfix{\beta}{\kind}{(\foralltinfix{\alpha}{\kind}{\semit{\Close}{\beta}})} & \foralltinfix{\beta}{\kind}{(\foralltinfix{\alpha}{\kind}{\semit{\Close}{\alpha}})}  \\
            \foralltinfix{\alpha}{\kind}{\semit{\Close}{\beta}} & \foralltinfix{\alpha}{\kind}{\semit{\Close}{\alpha}} \\
            \semit{\Close}{\beta} & \semit{\Close}{\alpha}\\
            \Skip & \Skip\\
        };
        % Draw arrows
        \draw[->] (m-1-1) -- (m-2-1) node[midway, left] {\symkeyword{\forall_0}};
        \draw[->] (m-1-2) -- (m-2-2) node[midway, right] {\symkeyword{\forall_0}};
        \draw[->] (m-2-1) -- (m-3-1) node[midway, left] {\symkeyword{\forall_1}};
        \draw[->] (m-2-2) -- (m-3-2) node[midway, right] {\symkeyword{\forall_1}};
        \draw[->] (m-3-1) -- (m-4-1) node[midway, left] {\symkeyword{close}};
        \draw[->] (m-3-2) -- (m-4-2) node[midway, right] {\symkeyword{close}};
    \end{tikzpicture}
\end{center}


\LIMPA



