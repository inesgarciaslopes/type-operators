\chapter{Introduction}

O relatório final deverá ter, em geral, entre 50 e 90 páginas (sem considerar anexos). O seu conteúdo deve realçar o trabalho realizado pelo aluno e a sua contribuição concreta no trabalho. Por exemplo, se o trabalho consiste no desenvolvimento de vários módulos a serem integrados num sistema mais global, o aluno deverá preocupar-se em descrever a parte que desenvolveu, como desenvolveu, que ferramentas usou, que alternativas poderiam existir, etc., em vez de efectuar uma descrição exaustiva das funcionalidades de todo o sistema.

O número de capítulos no relatório final não é rígido. No entanto, recomenda-se que sejam adoptados os seguintes princípios para a organização do relatório:

\begin{enumerate}
\item Um capítulo \emph{introdutório} no qual se apresentam o contexto do trabalho, se resume o trabalho desenvolvido, se identificam as contribuições deste e se apresenta a estrutura do próprio relatório. Deverá também ser mencionado sucintamente o enquadramento institucional em que o trabalho decorreu.
\item Um capítulo no qual se apresentam \emph{em pormenor} os \emph{objectivos} do trabalho, o \emph{contexto subjacente}, a \emph{metodologia} utilizada no seu desenvolvimento bem como o \emph{planeamento} efectuado para o concretizar. Deve também ser apresentada uma confrontação com o plano de trabalho inicial analisando as razões de eventuais desvios ocorridos.
\item Um capítulo onde é descrito o \emph{trabalho realizado}. Este é um dos capítulos fundamentais do relatório. Apresenta concretamente o que se fez de facto e as ferramentas usadas. De notar que é importante que fique claro qual a contribuição concreta do trabalho, sobretudo em casos de trabalho em equipa. Neste capítulo poderão ser inseridas questões relevantes da área de estudo em que o trabalho se integra, assim como o possível enquadramento num trabalho mais amplo. Eventualmente, e em função do âmbito e dimensão do trabalho, este capítulo poderá ser substituído por um conjunto de outros capítulos, que englobem em si o \emph{trabalho relacionado}, a \emph{análise} do problema, o \emph{desenho} da solução, a \emph{implementação} da solução e a \emph{avaliação} desta.
\item Um capítulo no qual são apresentadas as \emph{conclusões}. Para além de um \emph{sumário} do trabalho realizado, deve ser feito um \emph{comentário crítico} e serem apresentadas possibilidades de \emph{trabalho futuro} referindo o que falta fazer e o que poderá ser melhorado.
\item Um capítulo com a \emph{bibliografia} - lista de documentos usados e outras referências consideradas relevantes.
\item Um conjunto de capítulos com os \emph{anexos}. Quaisquer listagens, informação confidencial ou outras descrições muito pormenorizadas não devem ser integradas no corpo principal do relatório. Se houver necessidade de as apresentar, sugere-se a sua introdução em anexos ou em documentos separados. Os anexos suplementam o relatório e como tal devem ser referidos no corpo principal do relatório, descrevendo o tipo de informação que se detalha em anexo.
\end{enumerate}

Quando concluir a sua Dissertação, ou Relatório Final, o aluno deverá entregar, no Gabinete de Estudos Pós-Graduados da FCUL, o seguinte:

\begin{itemize}
\item Requerimento de admissão a provas de Mestrado (ver secção 4.2 do Guia de PEI);
\item 7 exemplares da Dissertação, ou Relatório Final, (encadernados de forma a que seja possível escrever na lombada - não utilizar argolas);
\item 7 Curricula Vitae;
\item 3 CDs com a Dissertação, ou Relatório Final, gravado em formato PDF;
\item Parecer do orientador do DI sobre a Dissertação, ou Relatório Final, (ver secção 4.2 do Guia de PEI).
\end{itemize}

O aluno deverá também submeter a versão final, em formato PDF, da Dissertação, ou Relatório Final, através do PEIpal. Este documento não deverá, em condições normais, exceder os 5MiB (se isso acontecer então deve ser revista a qualidade das imagens evitando a inclusão de bitmaps).

Se houver fundamentação adequada, os relatórios de trabalho poderão ser escritos em Inglês. Para isso o aluno deverá entregar no Gabinete de Estudos Pós-Graduados da FCUL:

\begin{itemize}
\item Um pedido dirigido ao Presidente do Conselho Científico da FCUL, fundamentando a necessidade da escrita do relatório em Inglês (ver secção 4.2 do Guia de PEI);
\item Um parecer do orientador indicando que concorda com o pedido do aluno e, eventualmente, apresentando argumentos adicionais (ver secção 4.2 do Guia de PEI).
\end{itemize}

Tendo sido aceite a escrita em Inglês do relatório de trabalho, este deverá conter um resumo adicional em Português de, pelo menos, 1200 palavras.

\section{Background and Motivation}
Exploring sophisticated type systems and their seamless integration into programming languages is a thoroughly researched field. From System $\FMu$ \cite{GauthierP04} up to System $\FMuOmega$ \cite{DBLP:conf/popl/CaiGO16}, how far can we go until these systems are no longer suitable for compilers.

In the development of modern type systems, combining advanced features such as equirecursion, higher-order polymorphism, and higher-order context-free session types presents unique challenges and opportunities. The primary motivation for our research is to integrate these elements into a cohesive type system that can be practically incorporated into programming languages. Therefore, we are interested in practical algorithms for type equivalence checking to be incorporated into compilers.


\section{Objectives}


\section{Challenges}

\section{Contributions}

\section{Thesis Structure}

Este documento está organizado da seguinte forma:
\begin{itemize}
\item Capítulo 2 – AAA
\item Capítulo 3 – BBB
\end{itemize}

