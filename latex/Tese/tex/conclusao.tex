% Conclusao

\chapter{Conclusão}

%The evolution from System F to System Fμ and finally to System Fμ; with Session Types illustrates the increasing complexity and expressiveness of type systems. 

%**System F** introduces **parametric polymorphism**, enabling the creation of generic, reusable functions and data structures.
%System Fμ** adds **recursive types**, allowing for self-referential types and the definition of complex, recursive data structures. System Fμ;** extends this framework with **session types**, ensuring that communication between concurrent processes adheres to a predefined protocol, with recursion allowing for the modeling of complex, iterative communication patterns.

%This progression highlights how type systems can be enriched to handle not just data structures, but also communication protocols, providing a robust foundation for ensuring correctness in both sequential and concurrent programming.

To summarize, we investigated the integration of  $F^{\mu_*;}_\omega$ with context-free session types into the functional programming language FreeST. Context-free session types enhance the expressiveness and adaptability of communication protocols in programming languages, surpassing the limitations of regular session types. 

Our research tackled the significant challenges posed by type equivalence algorithms within this advanced type system. We emphasized the importance of handling recursive types separately to ensure the termination of normalisation. By refining reduction rules and employing a pre-kinding approach, type formation is decidable.

By reducing the problem to the bisimilarity of simple grammars, a robust solution for type equivalence checking is met, facilitating the implementation of advanced type systems in real-world programming languages.
\medskip

\paragraph{Acknowledgements.}
  Support for this research was provided by the Fundação para a Ciência e a
  Tecnologia through project SafeSessions ref.\ PTDC/CCI-COM/6453/2020, and by the
  LASIGE Research Unit ref.\ UIDB/00408/2020
  % (https://doi.org/10.54499/UIDB/00408/2020)
  and UIDP/00408/2020.
  % (https://doi.org/10.54499/UIDP/00408/2020).
