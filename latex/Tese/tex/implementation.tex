\chapter{Type Operators in FreeST: Implementation}

\renewcommand{\arraystretch}{1.0}
\begin{table}[h!]
    \centering
    \begin{tabular}{| @{\hskip 0.1in}p{0.3\linewidth}@{\hskip 0.1in} | @{\hskip 0.1in}p{0.10\linewidth}@{\hskip 0.1in} | @{\hskip 0.1in}p{0.45\linewidth}|}
        \hline
        \textbf{Module name} & \textbf{LoC} & \textbf{Description}\\
        \hline
        Syntax & 118 & defines the Type data constructor as well as the higher-order kind system, based on \cref*{fig:syntax-types}.\\
        \hline
        Substitution & 50 & implements capture-avoiding substitution on types.\\
        \hline
        Normalisation & 80 & reduces types to weak head normal form, i.e., until no further reduction is possible.\\
        \hline
        TypeFormation & 58 & implements the type checking algorithm.\\
        \hline
        Rename & 30 & renames bound variables in a type by the smallest possible variable available, i.e., the first which is not free on the type.\\
        \hline
        WeakHeadNormalForm & 86 & checks whether a type is in weak head normal form.\\
        \hline
        Grammar & 74 & defines the Grammar data constructor, based on the definition found in \cref*{sec:deciding-type-equivalence}.\\
        \hline
        TypeToGrammar & 179 & implements the function $\word$, that converts types into simple grammars.\\
        \hline
    \end{tabular}
    \caption{Modules.}
    \label{tab:modules}
\end{table}

\todo{either table 4.1 here as an overview/introduction or at the end as a summary.}

\section{Types}

\begin{lstlisting}
module Syntax (Type(..), LabelMap, Polarity(..), Sort(..), View(..), Variable, Kind(..), BaseKind(..), dual) where
\end{lstlisting}

A \lstinline{LabelMap} consists on a map where the keys are Variables, i.e integers, and the values are Types.

A \lstinline{Polarity} defines whether a type has an output ($\OUTn$) functionality or input ($\INn$). 

A \lstinline{View} defines whether a choice is considered to be internal () or external ().

A \lstinline{Sort} defines whether a type is a record or a variant.

A \lstinline{BaseKind} can be either a session kind or a functional kind. 

The datatype \lstinline{Kind} tells us a kind can be a proper-kind, i.e a basekind, or constructed by an arrow, $k -> k$.

\todo{define used haskell syntax carefully}

\section{Reduction and Weak Head Normal Form}

\begin{lstlisting}
module WeakHeadNormalForm (isVar, head, isConstant, args, iswhnf) where
\end{lstlisting}

\begin{lstlisting}
    iswhnf :: Type -> Bool
\end{lstlisting}

\lstinline{iswhnf} takes a type and returns if it is in weak head normal form, according to the rules in fig.[]. This function depends on the functions isVar, isConstant, head, neck and depth, that we export with the module, as well as functions that verifiy the pre-conditions of each rule present in fig. [].

\begin{lstlisting}
module Normalisation (norm, weaknorm, red) where
\end{lstlisting}

\begin{lstlisting}
    norm :: Type -> Bool
    weaknorm :: MuSet -> Type -> Maybe Type
\end{lstlisting}

\lstinline{norm} returns true if a type normalises to a weak head normal form, otherwise returns false.

\lstinline{weaknorm} takes two arguments, a set (containing only mu types) and a type, returns the monadic Maybe of a type---Nothing if the type does not reduce to a weak head normal form, Just if the type reduces.

\todo{probably take this code out but explain carefully each step.}
\begin{lstlisting}
weaknorm :: MuSet -> Type -> Maybe Type
weaknorm visited t = 
  case reduceBSD t of
    Just t' -> weaknorm visited t'
    Nothing -> case t of
      (App (Rec _) _) -> weaknorm' visited t
      (App (App Semi u@(App (Rec _) _)) _) -> weaknorm' visited u
      (App Dual u@(App (Rec _) _)) -> weaknorm' visited u
      (App (App Semi (App Dual u@(App (Rec _) _))) _) -> weaknorm' visited u
      _ -> Just t
  where
    weaknorm' s t
      | t `Set.member` s = Nothing -- been here before
      | otherwise = 
        case reduceMu t of
          Just t' -> weaknorm (Set.insert t s) t'
          Nothing -> Just t
\end{lstlisting}

\begin{lstlisting}
module Rename (rename) where
\end{lstlisting}

\todo{talk about example that changed rename. }
\begin{lstlisting}
module Substitution (substitution, freeVars) where

\end{lstlisting}
\section{Type Formation}

\begin{lstlisting}
module TypeFormation (synthetise, preSynthetise -- tests only) where

\end{lstlisting}

\begin{lstlisting}
    synthetise :: KindingCtx -> Type -> Maybe Kind
\end{lstlisting}

synthetise takes a kinding context (a map Variable-Kind) a type and returns if the type is well-formed. It consists on two steps: first we check if the type is pre-kinded, where we invoce function \lstinline{preSynthetise}; then we check if T normalises. 

\todo{take figure where the rules are written and overlay in a darker grey which part represents each phase of synthetising a type.}
\todo{Another take would be do to a diagram: Is T a type? }

\subsection{Decidability of type formation}
\subsection{Restrictions to Recursion}
\todo{we keep a set of visited mu types, so we know when to terminate the algorithm.}

\section{Type equivalence + Decidability of type equivalence}
\begin{lstlisting}
module Grammar ( Grammar(..), Word, Label, Productions, Transitions) where
\end{lstlisting}

\begin{lstlisting}
module TypeToGrammar (word, convertToGrammar) where
\end{lstlisting}
\LIMPA