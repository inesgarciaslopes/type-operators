% !TeX root = samplepaper.tex

\begin{figure}[t]
  %
  \begin{align*}
	\tsymc a &\grmeq
    % \tsymc\alpha \grmor \tsymc\iota \grmor
     \tsymc{\alpha_i} \grmor \tsymc{\iota_i} \grmor
               \tsymc{\lambda\alpha\colon\kind}
  & (i\ge0, \tsymc\iota\neq\Skip)
  && \text{Transition labels}
  \end{align*}
  % 
  \declrel{Labelled transition system}{$\ltsred{T}{a}{U}$}
  %
  \begin{mathpar}
  \infer[\rulename{L-Red}]{
    \lambdared TU 
    \quad
    \ltsred{U}{a}{V}
  }{
    \ltsred{T}{a}{V}
  }
  
  \infer[\rulename{L-Var1}]{
    m\geq 0
  }{
    \ltsred{\alpha\ T_1\ldots T_m}{\alpha_0}\Skip
  }
  
  \infer[\rulename{L-Var2}]{
    1\leq j\leq m
  }{
    \ltsred{\alpha\ T_1\ldots T_m}{\alpha_j}{T_j}
  }
  
  \infer[\rulename{L-Const}]{
    \tiota \neq \Skip
  }{
    \ltsred\iota{\iota_0}\Skip
  }
  
  \infer[\rulename{L-ConstApp}]{
    \typec \iota= \function{}{}, \typec{\odot{\{\overline{\ell_i}\}}}, \typec{\varrecs{\overline{\ell_i}}}
    \quad
    1\leq j\leq m
  }{
    \ltsred{\iota\ T_1\ldots T_m}{\iota_j}{T_j}
  }
  
  \infer[\rulename{L-Abs}]{
    \ltsred{\tlambda \alpha\kind T}{\lambda\alpha\colon\kind}{T}
  }{}

  \infer[\rulename{L-Quant1}]{
    \ltsred{\quant\kind\ T}
    {\lquant\kind_1}{T}
  }{}
  
  \infer[\rulename{L-Quant2}]{
    \ltsred{\quant\kind\ T}
    {\lquant\kind_2}\Skip
    }{}

  \infer[\rulename{L-Msg1}]{
    \ltsred{\MSGn T}{\sharp_1}{T}
  }{}
  
  \infer[\rulename{L-Msg2}]{
    \ltsred{\MSGn T}{\sharp_2}{\Skip}
  }{}
  
  \infer[\rulename{L-Seq1}]{
    \ltsred{;\ T}{;_1}{T}
  }{}
  
  \infer[\rulename{L-VarSeq1}]{
    m\geq 0
  }{
    \ltsred{\semit{(\alpha\ T_1\ldots T_m)}U}{\alpha_0}{U}
  }
  
  \infer[\rulename{L-VarSeq2}]{
    1\leq j\leq m
  }{
    \ltsred{\semit{(\alpha\ T_1\ldots T_m)}U}{\alpha_j}{T_j}
  }

  \infer[\rulename{L-QuantSeq1}]{
        \ltsred{\semit{(\quant\kind\ T)}U}{\lquant\kind_1}{T}
  }{}
  
  \infer[\rulename{L-QuantSeq2}]{
        \ltsred{\semit{(\quant\kind\ T)}U}{\lquant\kind_2}{U}
  }{}
  
  \infer[\rulename{L-MsgSeq1}]{
    \ltsred{\semit{\MSGn T}U}{\sharp_1}{T}
  }{}
  
  \infer[\rulename{L-MsgSeq2}]{
    \ltsred{\semit{\MSGn T}U}{\sharp_2}{U}
  }{}
  
  \infer[\rulename{L-ChoiceSeq}]{
    1\leq j\leq m
  }{
    \ltsred{\semit{\choice\records{\overline{\ell_i\colon T_i}}}U}{\tsymc{\odot{\{\overline{\ell_i}\}}_j}}{\semit{T_j}U}
  }
  
  \infer[\rulename{L-WaitSeq}]{
    \ltsred{\semit\Wait U}{\keyword{Wait}}{\Skip}
  }{}

  \infer[\rulename{L-CloseSeq}]{
    \ltsred{\semit\Close U}{\keyword{Close}}{\Skip}
  }{}
  
  \infer[\rulename{L-DualVar1}]{
    \ltsred{\tdual{(\alpha\ T_1\ldots T_m)}}{\Duall_1}{\alpha\ T_1\ldots T_m}
  }{}
  
  \infer[\rulename{L-DualVar2}]{
    \ltsred{\tdual{(\alpha\ T_1\ldots T_m)}}{\Duall_2}{\Skip}
  }{}
  
  \infer[\rulename{L-DualSeq1}]{
    \ltsred{\semit{(\tdual{(\alpha\ T_1\ldots T_m)})} U}{\Duall_1}{\alpha\ T_1\ldots T_m}
  }{}
  
  \infer[\rulename{L-DualSeq2}]{
    \ltsred{\semit{(\tdual{(\alpha\ T_1\ldots T_m)})} U}{\Duall_2}{U}
  }{}
  \end{mathpar}
  \caption{Labelled transition system for types.}
  \label{fig:lts}
\end{figure}

%%% Local Variables:
%%% mode: latex
%%% TeX-master: "samplepaper"
%%% End:
