\section{System $F^{\mu_*;}_\omega$}\label{sec:system}
In programming languages, terms are categorized by types, which in turn may be categorized by kinds. In our system, a \emph{kind} $\kind$ is either a base kind $\kast$---which is either a session or a functional kind, $\skind$ or $\tkind$, respectively---or a higher-order kind  $\karrow \kind \kindp$. A \emph{proper type} refers to a type that has a base kind. In contrast, \emph{type operators} act upon types to create more complex types, and are associated with higher-order kinds. For example, the type operator $\Arrow$ takes two types, $\typec{T}$ and $\typec{U}$, and constructs a new type, a function type $\Arrow \typec{T} \typec{U}$, which we sometimes write in infix notation as $\typec{T}\ \Arrow\ \typec{U}$. This is the core of our work. Our goal is to expand the programming language FreeST, currently limited to types of kind $\kast$, to higher-order kinds.

%fix later.

\begin{figure}[t]
	\centering
	%\begin{minipage}{.45\textwidth}
		\begin{align*}
			\TT \grmeq &&&&&&&& \text{\small Type}\\
			& \tiota && \text{\small type constructor} && \tabs \alpha\kind T && \text{\small  type-level abstraction}\\
			& \talpha && \text{\small  type variable} && \tapp TT && \text{\small  type-level application}
        \end{align*}
        \begin{align*}
           \tiota \grmeq &&&&&&&&&&& \text{\small\qquad\quad {Type constructor}}\\
            % Functional
			&\Arrow && \karrow \kast {\karrow {\kast}{\tkind}} &&  \text{\small arrow} && \typec{\varrecs{\overline{l_i}}} && \kindc{\overline{{\kast \Rightarrow}}\,\tkind} && \text{\small record/variant}\\
			&\tmu \kind && \karrow {(\karrow \kind \kind)} {\kind} && \text{\small recursive type}
			&& \Sharp && \karrow \kast \skind &&\text{\small input/output} \\
            % Session types
            &\Semi && \karrow \skind {\karrow \skind \skind} && \text{\small seq.composition}
			 && \typec{\odot{\{\overline{l_i}\}}} && \kindc{\overline{{\skind
						\Rightarrow}}\,\skind} && \text{\small internal/external choice}\\
            &\quant \kind && \karrow {(\karrow \kind \kast)} \tkind && \text{\small exists/forall}
            && \Dual && \karrow{\skind}{\skind} && \text{\small dual operator}\\
            &\End_{\typec{\sharp}} && \skind && \text{\small wait/close} && \Skip && \skind && \text{\small skip} 
			%&\Skip && \skind && \text{\small skip} &&&
        \end{align*}
    \begin{align*}
    \typec{\varrecs{}} \grmeq{} \typec{\records{}} \!\grmor\! \typec{\variants{}}
    &&
    \typec\sharp \grmeq{} \typec? \!\grmor\! \typec!
    &&
    \choice \grmeq{} \intchoice \!\grmor\! \extchoice
    &&
    \quant\kind \grmeq{} \typec{\exists_\kind} \!\grmor\! \typec{\forall_\kind}
  \end{align*}
    %\end{minipage}
    \caption{The syntax of types and constructors.}
    \label{fig:syntax-types}
\end{figure}


%%% Local Variables:
%%% mode: latex
%%% TeX-master: "samplepaper"
%%% End:


A type is either a type constructor $\tiota$, a type variable $\talpha$, an abstraction $\tabs{\talpha}{\kind}{\typec{T}}$ or an application $\tapp{\typec{T}}{\typec{T}}$. A detailed list of types and constructors is in \cref{fig:syntax-types}. Observe that $\tmuinfix{\alpha}{\kind}{T}$ is syntactic sugar for $\typec{\tmu\kind (\tlambda{\alpha}{\kind}{T})}$ and similarly for $\tquantinfix\alpha\kind \TT$.

Not all types are of interest---for example $\tapp{\Dual}{\forallt{\skind}{\tabs{\alpha}{\skind}{\alpha}}}$ since the dual operator makes no sense being applied to a functional type. Before we can discuss type formation, we must define the weak head normal form of a type; we do so by defining a system of reduction rules in \cref{fig:reduction}. This system is that of Poças et al.~\cite{DBLP:conf/esop/PocasCMV23} made confluent by adding the proviso that $\typec{T} \neq \typec{\seqcomp{T_1}{T_2}}$ to rules \rseqtwo and \rdctx. \emph{Confluence} states that, if there are two distinct reductions for a given type, $\afunction{T}{U}$ and $\afunction{T}{V}$, then both paths will eventually converge into the same final reduced type $\typec{W}$. Our variant features a single reduction path, thus confluence immediately follows.
%With respect to Po\c{c}as et al. \cite{PocasCMV23}, we promoted polymorphism to kind $\skind$, which entails the introduction of the existential type $\texists \kind$, dual of $\tforall \kind$, and the corresponding reduction rules.

% !TeX root = samplepaper.tex
\begin{figure}[t]
	\declrel{Type reduction}{$\lambdared TT$}
	\begin{mathpar}
		\infer[\rseqone]{}{\lambdared{\semit \Skip T}{T}}
		
		\infer[\rseqtwo]{\lambdared{T}{V} \\ \typec{T} \neq \typec{T_1; T_2}}{\lambdared{\semit TU}{\semit VU}}
		
		\infer[\rassoc]{}{\lambdared{\semit {(\semit TU)}V}{\semit T {(\semit UV)}}}
		
		\infer[\rmu]{}{\lambdared{\tapp{\tmu{k}}\TT}{\tapp{T}({\tapp{\tmu{k}} \TT})}}

		\infer[\rbeta]{}{\lambdared{\tapp{(\tabs{\alpha}{\kind}{T})}{U}}{\TT\subs U\alpha}}

        \infer[\rabs]{\lambdared TU}{\lambdared{\tabs\alpha\kind T}{\tabs\alpha\kind U}}
		
		\infer[\rtappl]{\lambdared TU}{\lambdared{TV}{UV}}

		\infer[\rdsemi]{}{\lambdared{\tapp \Dual {(\semit T U)}}{\semit {\tapp \Dual
					T}{\tapp \Dual U}}}
		
		\infer[\rdskip]{}{\lambdared {\tapp \Dual \Skip}{\Skip}}
		
		%\infer[\rdend]{}{\lambdared {\tapp \Dual \End}{\End}}

        \infer[\rdwait]{}{\lambdared{\tapp \Dual \Wait}{\Close}}

        \infer[\rdclose]{}{\lambdared{\tapp \Dual \Close}{\Wait}}
		
		\infer[\rdin]{}{\lambdared{\tapp \Dual {(\tapp ?T)}}{\tapp !T}}
		
		\infer[\rdout]{}{\lambdared{\tapp \Dual {(\tapp !T)}}{\tapp ?T}}
		
		\infer[\rdextc]{}{\lambdared{\tapp {\Dual} {( {\&\{\overline{l_i:T_i}\}})}}{\oplus\{\overline{l_i:\Dual(T_i)}\}}}
		
		\infer[\rdintc]{}{\lambdared{\tapp {\Dual} {( {\oplus\{\overline{l_i:T_i}\}})}}{\&\{\overline{l_i:\Dual(T_i)}\}}}
  
		\infer[\rdctx]{\lambdared TU \\ \typec{T} \neq \typec{T_1;T_2}}{\lambdared{\tapp \Dual T}{\tapp \Dual U}}
		
		\infer[\rddvar]{}{\lambdared{\tapp {\Dual} {(\tapp \Dual (\alpha\ \TT_1\ldots \TT_m))}}{\alpha\ \TT_1\ldots \TT_m}}

        \infer[\rdexists]{}{\lambdared{\tapp {\Dual}{(\texists{\kind} \TT)}}{}}{\tforall{\kind} \TT}

        \infer[\rdforall]{}{\lambdared{\tapp {\Dual}{(\texists{\kind} \TT)}}{}}{\tforall{\kind} \TT}
	\end{mathpar}
	\caption{Type reduction.}
	\label{fig:reduction}
\end{figure}

%%% Local Variables:
%%% mode: latex
%%% TeX-master: "samplepaper"
%%% End:

A type is in \emph{normal form}, denoted $\iswhnf{T}$, if it has been completely reduced, \textit{i.e.}, no further reductions are possible. In other words, $\iswhnf{T}$ iff $\typec{T} \nrightarrow$. Then we can say that type $\typec{T}$ \emph{normalises} to type $\typec{U}$, written $\isnorm{T}{U}$, if $\iswhnf{U}$ and $\typec{U}$ is reached from $\typec{T}$ in a finite number of reduction steps. The predicate $\isnormalised{T}$ means that $\isnorm{T}{U}$ for some $\typec{U}$.
Note that not all types normalise, \ie some have an infinite sequence of reductions, such as $\TT = \typec{(\lambda x. xx)(\lambda x. xx)} \longrightarrow \typec{(\lambda x. xx)\subs{(\lambda x. xx)} {x}} = \TT \longrightarrow \typec{\cdots}$---which is stuck in a loop of reductions to itself via the application of rule \rbeta---and $\UT = \typec{\tmu\skind (\tlambda{\alpha}{\skind}{\alpha;\Skip})} \rightarrow (\tlambda{\alpha}{\skind}{\alpha;\Skip})\UT \rightarrow \typec{U;\Skip} \rightarrow_2 \typec{U;\Skip;\Skip} \rightarrow \typec{\cdots}$---which successively applies rules \rmu, \rbeta in combination with \rseqtwo, resulting in the unending addition of a trailing $\Skip$.

Finally, we introduce \emph{well-formed types}. We use $\judgementrelctx{\Delta}{\typec{T}}{\colon}{\kind}$ to denote that $\typec{T}$ is a \emph{well-formed type} with kind $\kind$ under the kinding context $\Delta$, a map from type variables to kinds. 
The kinds of constants can be found in \cref{fig:syntax-types}. A variable $\talpha$ has kind $\kind$ if $\tbind\alpha\kind\in{\Delta}$. An abstraction $\tabs\alpha \kind \TT$ has kind $\karrow \kind {\kind'}$ if $\TT$ is well-formed. Note that $\Delta + \tbind\talpha\kind$ in rule \ktabs represents updating the kind of the type variable $\talpha$ to a new kind $\kind$ in the context $\Delta$ if $\tbind\talpha\kindp\in\Delta$ for some $\kindp$, or storing the kind of $\talpha$ in the context $\Delta$ if otherwise. Finally, rule \ktapp states that an application $\tapp T U$ is well-formed if $\TT$ and $\UT$ are types and $\tapp TU$ normalises, that is, $\isnormalised {\tapp TU}$.
In \cref{sec:deciding-type-formation} we prove decidability of type formation, imposing a restriction to kind $\kast$ for recursive types, also adopted by Poças et al.~\cite{DBLP:conf/esop/PocasCMV23}.

% !TeX root = main.tex

\begin{figure}[t!]
  \declrel{Type formation}{$\istype \Delta T \kind$}
  \begin{mathpar}
    \infer[\kconst]{}{\istype \Delta \iota {\kind_\iota}}
    \quad
    \infer[\kvar]{\tbind\alpha\kind\in{\Delta}}
    {\istype \Delta \alpha \kind}
    \quad
    \infer[\ktabs]{\istype {\Delta+\tbind\alpha \kind} {T} {\kind'}}
    {\istype \Delta {\tabs\alpha \kind \TT} {\karrow \kind {\kind'}}}
    \quad
    \inferrule[\ktapp]{\istype \Delta {T} {\karrow \kind {\kind'}} \;\;\, \istype \Delta U \kind \;\;\, \isnormalised {\tapp TU}}
    {\istype \Delta {\tapp T U} {\kind'}}
  \end{mathpar}
  \caption{Type formation.}
  \label{fig:type-formation}
\end{figure}

%%% Local Variables:
%%% mode: latex
%%% TeX-master: "main"
%%% End:


% \paragraph{Type equivalence.}
Type equivalence allows us to check whether two types, even if syntactically different, correspond to the same protocol. 
It is expected that two types that are alpha-congruent are equivalent, like for example $\tlambda{\alpha}{\kind}{\alpha}$ and $\tlambda{\beta}{\kind}{\beta}$. However, the task of checking if two types are equivalent may involve substitutions on-the-fly as one crosses along the types. We will avoid this by performing a renaming operation once on both types, right at the beginning of the type equivalence checking process. We follow the notion of renaming in \cite{DBLP:conf/esop/PocasCMV23}.

%Po\c{c}as et al \cite{PocasCMV23} propose three possible ways to check whether two (renamed) types are equivalent:
%\begin{itemize}
%    \item by using a set of coinductive rules;
%    \item by means of a bisimulation on types, built from a labelled transition system;
%    \item by translating types to grammars and checking bisimulation of grammars.
%\end{itemize}
%
%We focus on the third approach, which we detail in \cref{sec:deciding-type-equivalence}.