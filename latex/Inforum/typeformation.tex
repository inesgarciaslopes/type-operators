\section{Deciding Type Formation}\label{sec:deciding-type-formation}
The rules for type formation in \cref{fig:type-formation} involve determining if an application type $\TT\UT$ normalises.
Po\c{c}as et al. \cite{PocasCMV23} propose a two-step solution to this problem.
The first stage is the introduction of the concept of \emph{pre-kinding}. We denote this as $\prekind \Delta T \kind$, that is, $\typec{T}$ is pre-kinded with kind $\kind$ under the kinding context $\Delta$. The rules for pre-kinding are in \cref{fig:pre-kinding}. They differ from the rules for type formation in that, in rule \pktapp, there is no verification of the normalisation of $\TT\UT$. Pre-kinding excludes some (but not all) types that do not normalise, as is the case of $\typec{(\tlambda{\alpha}{\kind}{\alpha \alpha})(\tlambda{\alpha}{\kind}{\alpha \alpha})}$.

% !TeX root = samplepaper.tex

\begin{figure}[t]
  \declrel{Pre-kinding}{$\prekind \Delta T \kind$}
  \begin{mathpar}
    \infer[\pkconst]{}
    {\prekind \Delta \iota {\kind_\iota}}
    \quad
    \infer[\pkvar]{\tbind\alpha\kind\in{\Delta}}{\prekind \Delta \alpha \kind}
    \quad
    \infer[\pktabs]{\prekind {\Delta+\tbind\alpha \kind} {T} {\kind'}}{\prekind \Delta {\tabs\alpha \kind \TT} {\karrow \kind {\kind'}}}
    \quad
    \inferrule[\pktapp]{\prekind \Delta {T} {\karrow \kind {\kind'}} \;\;\, \prekind \Delta U \kind}{\prekind \Delta {\tapp T U} {\kind'}}
  \end{mathpar}
  \caption{Pre-kinding.}
  \label{fig:pre-kinding}
\end{figure}

%%% Local Variables:
%%% mode: latex
%%% TeX-master: "samplepaper"
%%% End:


\begin{figure}[t]
	\begin{equation*}
		\inferrule*[rightstyle = {\scriptsize \sc}, right = PK-TApp]{
			\inferrule*[rightstyle = {\scriptsize \sc}, right = PK-TAbs]{
				\inferrule*[rightstyle = {\scriptsize \sc}, right = PK-TApp, rightskip=9em]{
					\infer[\textsc{\scriptsize PK-Var}]{
						\kind = \karrow {\kind''} {\kind'}
					}{
						\prekind {\typec{\alpha}:\kind} {\alpha} {\karrow {\kind''} {\kind'}}
					}
					\quad
					\infer[\textsc{\scriptsize PK-Var}]{
						\bot\quad (\kind \neq \kindc{\kind''})}{
						\prekind {\typec{\alpha}:\kind} {\alpha} {\kind''}
					}
				}{
					\prekind {\typec{\alpha}:\kind} {\alpha \alpha} {\kind'}
				}
			}
			{
				\prekind {} {\tlambda{\alpha}{\kind}{\alpha \alpha}} {\karrow {\kind} {\kind'}}
			}
			\qquad
			\inferrule*[]{
				\vdots}{
				\prekind {} {\tlambda{\alpha}{\kind}{\alpha \alpha}} {\kind}
			}
		}
		{
			\prekind {} {(\tlambda{\alpha}{\kind}{\alpha \alpha})(\tlambda{\alpha}{\kind}{\alpha \alpha})} {\kind'}
		}
	\end{equation*}
	\caption{Example of an unsuccessful derivation $\prekind {} {(\tlambda{\alpha}{\kind}{\alpha \alpha})(\tlambda{\alpha}{\kind}{\alpha \alpha})} {\kind'}$.}
	\label{fig:ex-pre-kinding}
\end{figure}

For a type which is pre-kinded, termination of $\isnormalised {\TT}$ is guaranteed. Some recursive types are problematic for normalisation, as the application of reduction might not decrease their size. For example, the type $\tmu\skind\ (\tabs\alpha\skind {\alpha;\Skip})$ is pre-kinded but successive reduction steps---via \rmu and \rbeta---keep adding $\typec{;}\Skip$ to the tail of the type so we must conclude that it does not normalise.
When dealing with normalisation, we separate the treatment of recursive types from the remaining types.
In particular, we divide the reduction rules in two groups: $\muarrow$ refers to reductions that use the $\rmu$ rule and $\lseqarrow$ refers to reductions that never invoke the $\rmu$ rule. Thus, $\betaarrow\ =\ \lseqarrow\cup\muarrow$. We may now lift this notion to normalisation, denoted by $\lseqnormalred TU$ and $\munormalred TU$ respectively.

In order to check if a type $\TT$ is well-formed, we first determine if $\prekind{} T \kind$ for some $\kind$. If $\TT$ fails to be pre-kinded, it is not kinded either. Otherwise, we check whether $\istype{} T \kind$, which involves determining if the application types within $\TT$ normalise. The approach to determine if a type normalises seeks infinite reduction sequences. In the case of recursive types, such sequences would have a finite number of $\beta$-reductions between two $\mu$-reductions.
$\TT = \lseqnormalred{\typec{T_0}}{\typec{T'_0}}\mured{}{\typec{T_1}}\lseqnormalred{}{\typec{T'_1}}\mured{}{\typec{T_2}}\lseqnormalred{}{\typec{T'_2}}\mured{}{\cdots}$.
If $\typec{T'_i}$ does not reduce by any $\mu$-reduction, we can conclude that
$\typec{T}$ normalises. Otherwise, since $\tmu\kast \UT$ is restricted to a base
kind $\kast$, it must reduce by one of following cases.
\begin{align*}
\typec{T'_i} &= \mured{\tapp{\tmu\kast}\UT}{\tapp\UT{(\tapp{\tmu\kast}\UT)}} & (\rmu)
\\
\typec{T'_i} &= \mured{\semit{(\tapp{\tmu\kast}\UT)}{V}}{\semit{(\tapp U{(\tapp{\tmu\kast}\UT)})}{V}} & (\rseqtwo)
\\
\typec{T'_i} &= \mured{\tdual{(\tapp{\tmu\kast}\UT)}}{\tdual{(\tapp U {(\tapp{\tmu\kast}\UT)})}} & (\rdctx)
\\
\typec{T'_i} &= \mured{\semit{(\tdual{(\tapp{\tmu\kast}\UT)})}{V}}{\semit{(\tdual{(\tapp U {(\tapp{\tmu\kast}\UT)})})}{V}} & (\rseqtwo + \rdctx)
\end{align*}
We can easily notice that expression $\tmu\kast\UT$ reappears after the $\mu$-reduction, indicating potential infinite sequences. We can detect these by tracking occurrences of $\tmu\kast\UT$ and halting if a repetition is found.
%The following pseudo-code illustrates the process for determining if $\typec{T}$ normalises: 
%
%\begin{lstlisting}
%    normalises(visited, t) = 
%        if reducesByBSD(t) then
%            normalises(visited, t') -- t -> t'
%        else if memberOf(t, visited) then
%            Nothing -- found an infinite sequence
%        else if reducesByMu(t) then
%            normalises(visited', t') -- update visited set with t
%        else t
%\end{lstlisting}
%%% Local Variables:
%%% mode: latex
%%% TeX-master: "42-CR"
%%% End:
