\renewcommand{\arraystretch}{1.0}
\begin{table}[h!]
    \centering
    \begin{tabular}{| @{\hskip 0.1in}p{0.3\linewidth}@{\hskip 0.1in} | @{\hskip 0.1in}p{0.10\linewidth}@{\hskip 0.1in} | @{\hskip 0.1in}p{0.45\linewidth}|}
        \hline
        \textbf{Module name} & \textbf{LoC} & \textbf{Description}\\
        \hline
        Syntax & 118 & defines the Type data constructor as well as the higher-order kind system, based on \cref*{fig:syntax-types}.\\
        \hline
        Substitution & 50 & implements capture-avoiding substitution on types.\\
        \hline
        Normalisation & 80 & reduces types to weak head normal form, i.e., until no further reduction is possible.\\
        \hline
        TypeFormation & 58 & implements the type checking algorithm.\\
        \hline
        Rename & 30 & renames bound variables in a type by the smallest possible variable available, i.e., the first which is not free on the type.\\
        \hline
        WeakHeadNormalForm & 86 & checks whether a type is in weak head normal form.\\
        \hline
        Grammar & 74 & defines the Grammar data constructor, based on the definition found in \cref*{sec:deciding-type-equivalence}.\\
        \hline
        TypeToGrammar & 179 & implements the function $\word$, that converts types into simple grammars.\\
        \hline
    \end{tabular}
    \caption{Modules.}
    \label{tab:modules}
\end{table}