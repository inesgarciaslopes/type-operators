\renewcommand{\arraystretch}{1.2}
\begin{table}[t!]
    \centering
    \begin{tabular}{@{\hskip 0.1in}p{0.25\linewidth}@{\hskip 0.1in} @{\hskip 0.1in}p{0.05\linewidth}@{\hskip 0.1in} @{\hskip 0.1in}p{0.46\linewidth}}
    % \begin{tabular}{| @{\hskip 0.1in}p{0.3\linewidth}@{\hskip 0.1in} | @{\hskip 0.1in}p{0.10\linewidth}@{\hskip 0.1in} | @{\hskip 0.1in}p{0.45\linewidth}|}
        \hline
        \textbf{Module name} & \RaggedLeft \textbf{LoC} & \textbf{Description}\\
        \hline
        Syntax & \RaggedLeft 118 & Defines the \lstinline|Type| data constructor as well as the higher-order kind system, based on \cref*{fig:syntax-types}.\\
        % \hline
        Substitution & \RaggedLeft 50 & Implements capture-avoiding substitution on types.\\
        % \hline
        Normalisation & \RaggedLeft 80 & Reduces types to weak head normal form, i.e., until no further reduction is possible.\\
        % \hline
        TypeFormation & \RaggedLeft 58 & Implements the type checking algorithm.\\
        % \hline
        Rename & \RaggedLeft 30 & Renames bound variables in a type by the smallest possible variable available, i.e., the first which is not free in the type.\\
        % \hline
        WeakHeadNormalForm & \RaggedLeft 86 & Checks whether a type is in weak head normal form.\\
        % \hline
        Grammar & \RaggedLeft 74 & Defines the \lstinline|Grammar| data constructor, based on the definition found in \cref*{sec:deciding-type-equivalence}.\\
        % \hline
        TypeToGrammar & \RaggedLeft 179 & Implements the function $\word$, that converts types into simple grammars.\\
        \hline
    \end{tabular}
    \caption{Haskell modules.}
    \label{tab:modules}
\end{table}

%%% Local Variables:
%%% mode: latex
%%% TeX-master: "42-CR"
%%% End:
