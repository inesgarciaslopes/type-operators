\section{Conclusion}\label{sec:conclusion}
\todo{Rewrite} To summarise, we explored the integration of $F^{\mu_*}_\omega$ with context-free session types into the functional programming language FreeST. Context-free session types enhance the expressiveness and adaptability of communication protocols in programming languages by extending beyond the limitations of regular session types. Our work addressed the significant challenges posed by type equivalence algorithms within this advanced type system. Our findings underscore the importance of handling recursive types separately to ensure termination of normalisation. By refining the reduction rules and employing a pre-kinding approach, we are able to decide type formation.
We proposed a bisimulation-based type equivalence method for System $F^{\mu_*;}_\omega$, demonstrating its practicality and efficiency in ensuring decidability. By reducing the problem to the bisimilarity of simple grammars, we provided a robust solution for type equivalence checking, facilitating the implementation of advanced type systems in real-world programming languages.

\medskip

\noindent \textbf{Acknowledgements.} Support for this research was provided by the\linebreak Fundação para a Ciência e a Tecnologia through project SafeSessions\linebreak ref.\ PTDC/CCI-COM/6453/2020, by the LASIGE Research Unit\linebreak ref.\ UIDB/00408/2020 (https://doi.org/10.54499/UIDB/00408/2020) and ref.\ UIDP/00408/2020 (https://doi.org/10.54499/UIDP/00408/2020).
