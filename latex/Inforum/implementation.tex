\section{Implementation and validation}\label{sec:implementation}
Our implementation, written in Haskell, consists on eight modules, as described in \cref{tab:modules}.
\renewcommand{\arraystretch}{1.0}
\begin{table}[h!]
    \centering
    \begin{tabular}{| @{\hskip 0.1in}p{0.3\linewidth}@{\hskip 0.1in} | @{\hskip 0.1in}p{0.10\linewidth}@{\hskip 0.1in} | @{\hskip 0.1in}p{0.45\linewidth}|}
        \hline
        \textbf{Module name} & \textbf{LoC} & \textbf{Description}\\
        \hline
        Syntax & 118 & defines the Type data constructor as well as the higher-order kind system, based on \cref*{fig:syntax-types}.\\
        \hline
        Substitution & 50 & implements capture-avoiding substitution on types.\\
        \hline
        Normalisation & 80 & reduces types to weak head normal form, i.e., until no further reduction is possible.\\
        \hline
        TypeFormation & 58 & implements the type checking algorithm.\\
        \hline
        Rename & 30 & renames bound variables in a type by the smallest possible variable available, i.e., the first which is not free on the type.\\
        \hline
        WeakHeadNormalForm & 86 & checks whether a type is in weak head normal form.\\
        \hline
        Grammar & 74 & defines the Grammar data constructor, based on the definition found in \cref*{sec:deciding-type-equivalence}.\\
        \hline
        TypeToGrammar & 179 & implements the function $\word$, that converts types into simple grammars.\\
        \hline
    \end{tabular}
    \caption{Modules.}
    \label{tab:modules}
\end{table}
\renewcommand{\arraystretch}{1.0}
\begin{table}[h!]
    \centering
    \begin{tabular}{| @{\hskip 0.1in}p{0.3\linewidth}@{\hskip 0.1in} | @{\hskip 0.1in}p{0.3\linewidth}@{\hskip 0.1in} | @{\hskip 0.1in}p{0.25\linewidth}|}
        \hline
        \textbf{Property} & \textbf{Description} & \textbf{Number of Tests} \\
        \hline
        prop\_rename & Validates node structure remains unchanged after renaming & Passed all\\
        \hline
        prop\_tree\_structure & Confirms renaming maintains tree structure & Passed all\\
        \hline
        prop\_rename\_idempotent & Checks that renaming twice is equivalent to once & Passed all\\
        \hline
        prop\_reduction\_preserves\linebreak\_renaming & If $\lambdared TU$ and $\TT = \rename(\TT)$ then $\UT = \rename(\UT)$ & Passed 23464\newline Discarded 200000\\
        \hline
        prop\_renaming\_preserves\linebreak\_alpha\_congruence & Ensures alpha congruent types remain congruent after renaming & Passed 5858\newline  Discarded 200000\\
        \hline
        prop\_renaming\_reflects\linebreak\_alpha\_congruence & If $\TT$ and $\UT$ are alpha-congruent, then $\rename(\TT) = \rename(\UT)$ & Passed 5858\newline  Discarded 200000\\
        \hline
        prop\_nf\_does\_not\_reduce & Ensures that types in normal form do not reduce further & Passed 2374\newline  Discarded 200000\\
        \hline
        prop\_reduced\_is\_not\_nf & If $\lambdared TU$ then $\judgementlabel{not}(\issnf T)$ & Passed 2374\newline  Discarded 200000\\
        \hline
        prop\_preservation & If $\istype\Delta T \kind$ and $\betared{T}{U}$, then $\istype \Delta U \kind$ & Passed 576\newline  Discarded 200000\\
        \hline
    \end{tabular}
    \caption{Tested properties with Quickcheck.}
    \label{tab:properties}
\end{table}

The current FreeST compiler features an algorithm for checking the bisimilarity of simple grammars, which we use in our tests. Our testing process takes a suite of randomly generated types---a small subset of FreeST's types, based on the syntax presented in \cref*{sec:system}---leveraging the Quickcheck library\cite{DBLP:conf/icfp/ClaessenH00} to ensure these types have specific properties. Nevertheless, formal proofs regarding decidability of type formation and equivalence can be found in \cite{PocasCMV23}. 

An arbitrary type generator is defined using the \textit{Arbitrary} typeclass, employing the \textit{frequency} function to generate type operators with specific probabilities. Variables are selected from a predefined range, abstractions are created by generating a variable, a kind, and a sub-type, and applications are formed by recursively generating two sub-types. The \textit{sized} function is used to control the size of the generated types, ensuring manageable recursion depth. For better statistics we ensure proper distribution of type constructors. The list of properties can be found in \cref{tab:properties}. A total of 200.000 tests were made for each property.

Data was collected on a machine equipped with an Apple M3 Pro and 18GB of RAM, and tested with Haskell's version 9.6.3. We run each property 10 times and the time presented in \cref{tab:properties} is the average except the best/worst cases. 

While randomly generated types facilitate a robust analysis, certain properties, such as the preservation property and bisimilarity of simple grammars, prove challenging to test comprehensively. This difficulty arises from the simplicity of our generator and the inherent probability that randomly generated test cases preserve such properties. Most of the time, generating a type $\TT$ that is well-formed and reduces to some $\UT$ is very hard. Therefore, most of the tests cases do not satisfy the predicate, $\istype\Delta T \kind$ and $\betared{T}{U}$, and Quickcheck ends up discarding 199.910 tests for the last two properties. To achieve better results, more complex generators tailored to specific properties would be required, though such generators are often challenging to design and implement. 
