\section{Implementation and validation}\label{sec:implementation}
TO-DO

\begin{itemize}
    \item Implementation in Haskell
    \item lines of code
    \item modules and exported functions
\end{itemize}


Our validation process uses a suite of randomly generated types, leveraging the Quickcheck library\cite{DBLP:conf/icfp/ClaessenH00} to ensure these types have specific properties. An arbitrary type generator is defined using the \textit{Arbitrary} typeclass, employing the \textit{frequency} function to generate type operators with specific probabilities. Variables are selected from a predefined range, abstractions are created by generating a variable, a kind, and a sub-type, and applications are formed by recursively generating two sub-types. The \textit{sized} function is used to control the size of the generated types, ensuring manageable recursion depth. For better statistics we ensure proper distribution of type constructors. We present next the list of properties tested with a \textit{maxSuccess} of 200.000 tests:

\renewcommand{\arraystretch}{1.0}
\begin{table}[h!]
    \centering
    \begin{tabular}{| @{\hskip 0.1in}p{0.3\linewidth}@{\hskip 0.1in} | @{\hskip 0.1in}p{0.3\linewidth}@{\hskip 0.1in} | @{\hskip 0.1in}p{0.25\linewidth}|}
        \hline
        \textbf{Property} & \textbf{Description} & \textbf{Number of Tests} \\
        \hline
        prop\_rename & Validates node structure remains unchanged after renaming & Passed all\\
        \hline
        prop\_tree\_structure & Confirms renaming maintains tree structure & Passed all\\
        \hline
        prop\_rename\_idempotent & Checks that renaming twice is equivalent to once & Passed all\\
        \hline
        prop\_reduction\_preserves\linebreak\_renaming & If $\lambdared TU$ and $\TT = \rename(\TT)$ then $\UT = \rename(\UT)$ & Passed 23464\newline Discarded 200000\\
        \hline
        prop\_renaming\_preserves\linebreak\_alpha\_congruence & Ensures alpha congruent types remain congruent after renaming & Passed 5858\newline  Discarded 200000\\
        \hline
        prop\_renaming\_reflects\linebreak\_alpha\_congruence & If $\TT$ and $\UT$ are alpha-congruent, then $\rename(\TT) = \rename(\UT)$ & Passed 5858\newline  Discarded 200000\\
        \hline
        prop\_nf\_does\_not\_reduce & Ensures that types in normal form do not reduce further & Passed 2374\newline  Discarded 200000\\
        \hline
        prop\_reduced\_is\_not\_nf & If $\lambdared TU$ then $\judgementlabel{not}(\issnf T)$ & Passed 2374\newline  Discarded 200000\\
        \hline
        prop\_preservation & If $\istype\Delta T \kind$ and $\betared{T}{U}$, then $\istype \Delta U \kind$ & Passed 576\newline  Discarded 200000\\
        \hline
    \end{tabular}
    \caption{Tested properties with Quickcheck.}
    \label{tab:properties}
\end{table}

Data was collected on a machine equipped with an Apple M3 Pro and 18GB of RAM, and tested with Haskell's version 9.6.3.

While randomly generated types facilitate a robust analysis, certain properties, such as the preservation property, prove challenging to test comprehensively. This difficulty arises from the simplicity of our generator and the inherent probability of randomly generated test cases preserving such properties. To achieve better results, more complex generators tailored to specific properties would be required, though such generators are often challenging to design and implement. 
