\section{Implementation and validation}\label{sec:implementation}
Our implementation, written in Haskell, consists on eight modules:
\begin{itemize}
    \item The \emph{Syntax} module (LoC 118) defines the Type data constructor as well as the higher-order kind system, based on \cref*{fig:syntax-types}. 
    \item The \emph{Substitution} module (LoC 50) implements capture-avoiding substitution on types.
    \item The \emph{Normalisation} module (LoC 80) reduces types to weak head normal form, i.e., until no further reduction is possible.
    \item The \emph{TypeFormation} module (LoC 58) implements the type checking algorithm.
    \item The \emph{Rename} module (LoC 30) renames bound variables in a type by the smallest possible variable available, i.e., the first which is not free on the type.
    \item The \emph{WeakHeadNormalForm} module (LoC 86) checks whether a type is in weak head normal form. 
    \item The \emph{Grammar} module (LoC 74) defines the Grammar data constructor, based on the definition found in \cref*{sec:deciding-type-equivalence}.
    \item The \emph{TypeToGrammar} module (LoC 179) implements the function $\word$, that converts types into simple grammars.
\end{itemize}

The current FreeST compiler features an algorithm for type equivalence, which we use to test the bisimilarity of simple grammars.

Our validation process uses a suite of randomly generated types--a small subset of FreeST's types, based on the syntax presented in \cref*{sec:system}--leveraging the Quickcheck library\cite{DBLP:conf/icfp/ClaessenH00} to ensure these types have specific properties. Nevertheless, formal proofs regarding decidability of type formation and equivalence can be found in \cite{PocasCMV23}. 

An arbitrary type generator is defined using the \textit{Arbitrary} typeclass, employing the \textit{frequency} function to generate type operators with specific probabilities. Variables are selected from a predefined range, abstractions are created by generating a variable, a kind, and a sub-type, and applications are formed by recursively generating two sub-types. The \textit{sized} function is used to control the size of the generated types, ensuring manageable recursion depth. For better statistics we ensure proper distribution of type constructors. The list of properties can be found in \cref{tab:properties}. A total of 200.000 tests were made for each property.
\renewcommand{\arraystretch}{1.0}
\begin{table}[h!]
    \centering
    \begin{tabular}{| @{\hskip 0.1in}p{0.3\linewidth}@{\hskip 0.1in} | @{\hskip 0.1in}p{0.3\linewidth}@{\hskip 0.1in} | @{\hskip 0.1in}p{0.25\linewidth}|}
        \hline
        prop\_nf\_does\_not\_reduce & Ensures that types in normal form do not reduce further & Passed 2374\newline Time x.\\
        \hline
        prop\_reduced\_is\_not\_nf & If $\lambdared TU$ then $\judgementlabel{not}(\issnf T)$ & Passed 2374\newline Time x.\\
        \hline
        prop\_preservation & If $\istype\Delta T \kind$ and $\betared{T}{U}$, then $\istype \Delta U \kind$ & Passed 576\newline Time x.\\
        \hline
        prop\_bisimilar & If $\istype\Delta T \kind$ and $\betared{T}{T}$, then $\word(T)\sim\word(U)$ & Passed 90\newline Time x.\\
        \hline
    \end{tabular}
    \caption{Tested properties with Quickcheck.}
    \label{tab:properties}
\end{table}

Data was collected on a machine equipped with an Apple M3 Pro and 18GB of RAM, and tested with Haskell's version 9.6.3. We run each property 10 times and the time presented in \cref{tab:properties} is the average of the sum of each timer minus the best/worst scenarios. 

While randomly generated types facilitate a robust analysis, certain properties, such as the preservation property and bisimilarity of simple grammars, prove challenging to test comprehensively. This difficulty arises from the simplicity of our generator and the inherent probability of randomly generated test cases preserving such properties. To achieve better results, more complex generators tailored to specific properties would be required, though such generators are often challenging to design and implement. 
