\section{Deciding Type Equivalence}\label{sec:deciding-type-equivalence}
Following Poças et al. \cite{PocasCMV23}, the problem of checking whether two (renamed) types are equivalent is reduced to translating types into grammars and checking bisimilarity. A grammar in \emph{Greibach normal form} \cite{AutebertG84} is a tuple $(\mathcal{T, N}, \ntgamma, \mathcal{R})$, where:
\begin{itemize}
	\item $\mathcal{T}$ is a finite set of terminal symbols, $\tsymc{a}$, $\tsymc{b}$, $\tsymc{c}$; 
	\item $\mathcal{N}$ is a finite set of non-terminal symbols, $\Xnt, \Ynt, \Znt$;
	\item $\ntgamma\in\mathcal{N}^*$ is the starting word;
	\item $\mathcal{R} \subseteq \N \times \T \times \N^*$ is a finite set of production rules.
\end{itemize}

% !TeX root = samplepaper.tex
\begin{figure}[t]
    \declrel{}{$\word(\typec{T})$}
    \begin{align*}
        \word (\TT) = \begin{cases}
                        \word'(\TT) & \iswhnf \TT\\
                        \emptyword & \normalred T\Skip\\
                        \Ynt, \R := \R \cup \{\gltsred Y a {\gamma\delta} \ |\ \tsymc a, \nsymc \gamma : \gltsred Z a {\gamma}\} & \normalred TU \neq \Skip, \word(\UT) = \Znt \ntdelta
                    \end{cases}
    \end{align*}
  \begin{align*}
    \word'(\talpha\ \typec{T_1}\typec{\ldots}\typec{T_m}) &= \Ynt, \R := \R \cup \{\gltsred Y{\alpha_0}{\emptyword}, \gltsred Y{\alpha_j}{\wordb{\typec{T_j}}\bot}\} 
    \\
    \word'(\Skip) &= \emptyword
    \\
    \word'(\End_{\typec{\sharp}}) &=  \Ynt, \R := \R \cup \{\gltsred Y{\tsymc{\Endl}}{\bot}\}
%    \\
%    \word'(\Close) &= \Ynt, \R := \R \cup \{\gltsred Y\Closel{\bot}\}
    \\
    \word'(\tlambda{\alpha}\kind T) &= \Ynt, \R := \R \cup \{\gltsred Y {\lambda\alpha\colon\kind}{\wordb{\TT}}\}
    \\
    \word'(\tiota) &= \Ynt, \R := \R \cup \{\gltsred Y\iota\emptyword\} \quad
                     \text{where } \typec{\iota}\neq \Skip,\End_{\typec{\sharp}}
    \\
    \word'(\typec{\iota\ T_1\cdots T_m}) &= \Ynt, \R := \R \cup \{\gltsred Y{\iota_j}{\wordb{\typec{T_j}}}\} \quad \text{where }\typec{\iota} = \function{}{}, \typec{\choice\{\overline{l_i}\}}, \typec{\varrecs{\overline{l_i}}}
    \\
    \word'(\MSGn\TT) &= \Ynt, \R := \R \cup \{\gltsred Y{\tsymc{\sharp_1}}{\wordb{\TT}\bot}, \gltsred Y{\sharp_2}{\emptyword}\}
    \\
    \word'(\quant\kind\TT) &= \Ynt, \R := \R \cup \{\gltsred Y{\tsymc{\lquant\kind}}{\wordb{\TT}\bot}
%    , \gltsred Y{\lquant\kind_2}{\emptyword}
    \}
    \\
    \word'(\Semi\ \TT) &= \Ynt, \R := \R \cup \{\gltsred Y{;_1}{\wordb{\TT}}\}
    \\
    \word'(\semit \TT \UT) &= \word(\TT)\word(\UT)
    \\
    \word'(\tdual{(\talpha\ \typec{T_1}\typec{\ldots}\typec{T_m})}) &= \Ynt, \R := \R \cup \{\gltsred Y{\Duall_1}{\wordb{\talpha\ \typec{T_1}\typec{\ldots}\typec{T_m}}}, \gltsred Y{\Duall_2}\emptyword\}
  \end{align*}
  $\Ynt$ is a fresh non-terminal symbol in all cases,\\ 
  $\nsymc \epsilon$ is the empty word,\\
  $\nsymc \bot$ is a non-terminal symbol without productions.\\
  \caption{Function word($\TT$).}
  \label{fig:word}
\end{figure}


%%% Local Variables:
%%% mode: latex
%%% TeX-master: "42-CR"
%%% End:

A production rule in $\mathcal{R}$ is written as $\ltsred{\Xnt}{a}{\ntdelta}$. Grammars in GNF are \emph{simple} when, for every non-terminal symbol $\Xnt$ and every terminal symbol $\tsymc{a}$, there is at most one production rule $\ltsred{\Xnt}{a}{\ntdelta}$ \cite{KorenjakH66}.

The function $\word({\typec{T}})$, described in \cref{fig:word}, translates types to words of non-terminal symbols. If a type $\TT$ is in weak head normal form, the construction of $\word(\TT)$ updates the set of productions of $\TT$, according to one of the cases found in $\word'$. If $\TT$ is not in weak head normal form and normalises to $\Skip$, $\word(\TT)$ returns the empty word; otherwise, if there exists a type $\UT\neq \Skip$ such that $\TT$ normalises to $\UT$, $\word(\UT) = \Znt \ntdelta$ and $\Ynt$ a fresh new terminal, then for each production of $\Znt$ of the form $\gltsred Z a {\gamma}$, $\Ynt$ has a production of the form $\gltsred Y a {\gamma\delta}$. 
The application of the $\word$ function to a type $\TT$ terminates producing a simple grammar. This is only possible because our well-formed types normalise, and all of its subterms normalise as well. Furthermore, we keep track of already visited types which enable reusing non-terminal symbols, which is crucial for dealing with recursive types.

We check whether two types are equivalent by translating the (renamed) types to a simple grammar, and then checking their bisimilarity, \ie if $\word({\typec{T}}) \gequiv \word({\typec{U}})$. The algorithm used to check bisimilarity of simple grammars is in \cite{AlmeidaMV20}.

Consider the type $
\typec{T_0} = \tabs{\beta}{\tkind}{\tmuinfix{\alpha}{\skind}{\extchoice\rchannel{\leafl}{\Skip}{\nodel}{\alpha\Semi\ \INp\beta\Semi\ \alpha}}\Semi\Wait}$ described in \cref{sec:intro}. We will demonstrate how the construction of $\word(\typec{T_0})$ terminates generating a simple grammar.
Since $\typec{T_0}$ is in weak head normal form, $\word(\typec{T_0})$ returns a fresh symbol, which we call $\nsymc{X_0}$. We also add to the set of productions the production $\gltsred{X_0}{\lambda\beta\colon\tkind}{\wordb{\typec{T_1}}}$, where
$\typec{T_1}$ is the type $\tmuinfix{\alpha}\skind{\extchoice\records{\leafl\colon\Skip,\nodel\colon\semit{{\alpha}}{\semit{\INn{\beta}}{{\alpha}}}}\Semi\Wait}$.

Now $\typec{T_1}$ is not in weak head normal form, so we must normalise it in order to obtain $\typec{T_2}$ such that $\normalred {T_1} {T_2}$. Then, $\word(\typec{T_1})$ returns a fresh non-terminal which we call $\nsymc{X_1}$. To obtain the productions of $\typec{T_1}$, we need to compute $\word(\typec{T_2})$, that returns a fresh symbol $\nsymc{X_2}$. Since $\typec{T_2} = \typec{\extchoice\records{\leafl\colon\Skip,\nodel\colon\semit{T_1}{\semit{\INn {\beta}}{T_1}}}\Semi\Wait}$ is in weak head normal form, we need to first compute $\wordb{\typec{T_2}} = \word({\typec{T_3}})\word({\Wait})$, where $\typec{T_3}= \typec{\extchoice\records{\leafl\colon\Skip,\nodel\colon\semit{T_1}{\semit{\INn {\beta}}{T_1}}}}$. We have that $\wordb{\Wait} = \nsymc{X_4}$ and $\gltsred{X_4}{\tsymc{\waitl}}{\bot}$ but we still need to compute $\wordb{\typec{T_3}}$. This computation results in a fresh non-terminal $\nsymc{X_3}$ with productions $\gltsred{X_3}{\&_1}{\wordb{\Skip}}$ and $\gltsred{X_3}{\&_2}{\wordb{\semit{T_1}{\semit{\INn{\beta}}{T_1}}}}$. Therefore, we the transitions for $\nsymc{X_2}$ are $\gltsred{X_2}{\&_1}{\nsymc{X_4}}$ and $\gltsred{X_2}{\&_2}{\nsymc{X_3}\nsymc{X_4}}$.

At last, we must compute $\word(\semit{T_1}{\semit{\INn{\beta}}{T_1}})$, which is a fresh symbol $\nsymc{X_5}$, because this type is not in weak head normal form. This type normalises to $\semit{T_2}{\semit{\INn{\beta}}{T_1}}$, since $\normalred{T_1}{T_2}$, therefore the productions of $\nsymc{X_5}$ are the concatenation of $\word(\typec{T_2})\word(\INn{\beta})\word(\typec{T_1})$. At this point, we know that $\word(\typec{T_2})=\nsymc{X_2}$ and $\word(\typec{T_1})=\nsymc{X_1}$. Thus, we just need to compute $\word(\INn{\beta})$, which is a fresh symbol $\nsymc{X_6}$ with productions $\gltsred{X_6}{?_1}{\wordb{\typec{\beta}}\bot}$ and $\gltsred{X_6}{?_2}{\emptyword}$. Finally, $\word(\typec{\beta})$ is a fresh symbol $\nsymc{X_7}$ with a production $\gltsred{X_7}{{\beta}}{\emptyword}$. This means that $\word(\semit{T_2}{\semit{\INn{\beta}}{T_1}}) = \nsymc{X_2X_6X_1}$, which we write as $\gltsred{X_5}{\&_1}{X_4X_6X_1}$ and $\gltsred{X_5}{\&_2}{X_3X_4X_6X_1}$.

Putting everything together, we obtain the following simple grammar:
%
\begin{align*}
\gltsred{X_0}{\lambda{\beta}\colon\tkind}{X_1}
&&
\gltsred{X_1}{\&_1}{X_4}
&&
\gltsred{X_1}{\&_2}{X_3X_4}
&&
\gltsred{X_2}{\&_1}{X_4}
\\
\gltsred{X_2}{\&_2}{X_3X_4}
&&
\gltsred{X_3}{\&_1}{\emptyword}
&&
\gltsred{X_3}{\&_2}{X_5}
&&
\gltsred{X_4}{\tsymc{\waitl}}{\bot}
\\
\gltsred{X_5}{\&_1}{X_4X_6X_1}
&&
\gltsred{X_5}{\&_2}{X_3X_4X_6X_1}
&&
\gltsred{X_6}{?_1}{X_7\bot}
&&
\gltsred{X_6}{?_2}{\emptyword}
&& 
\gltsred{X_7}{\beta}{\emptyword}
&&
\end{align*}

