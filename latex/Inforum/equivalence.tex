\section{Dedicing type equivalence}\label{sec:deciding-type-equivalence}
Following Poças et al \cite{PocasCMV23}, the problem of checking whether two (renamed) types are equivalent is reduced to translating types into grammars and checking bisimilarity there. We say that a grammar in \emph{Greibach normal form} \cite{AutebertG84} is a tuple of the form $(\mathcal{T, N}, \ntgamma, \mathcal{R})$, where:
\begin{itemize}
	\item $\mathcal{T}$ is a finite set of terminal symbols, $\tsymc{a}$, $\tsymc{b}$, $\tsymc{c}$; 
	\item $\mathcal{N}$ is a finite set of non-terminal symbols, $\Xnt, \Ynt, \Znt$;
	\item $\ntgamma\in\mathcal{N}^*$ is the starting word;
	\item $\mathcal{R} \subseteq \N \times \T \times \N^*$ is a finite set of productions.
\end{itemize}

% !TeX root = samplepaper.tex
\begin{figure}[t]
    \declrel{}{$\word(\typec{T})$}
    \begin{align*}
        \word (\TT) = \begin{cases}
                        \word'(\TT) & \iswhnf \TT\\
                        \emptyword & \normalred T\Skip\\
                        \Ynt, \R := \R \cup \{\gltsred Y a {\gamma\delta} \ |\ \tsymc a, \nsymc \gamma : \gltsred Z a {\gamma}\} & \normalred TU \neq \Skip, \word(\UT) = \Znt \ntdelta
                    \end{cases}
    \end{align*}
  \begin{align*}
    \word'(\talpha\ \typec{T_1}\typec{\ldots}\typec{T_m}) &= \Ynt, \R := \R \cup \{\gltsred Y{\alpha_0}{\emptyword}, \gltsred Y{\alpha_j}{\wordb{\typec{T_j}}\bot}\} 
    \\
    \word'(\Skip) &= \emptyword
    \\
    \word'(\End_{\typec{\sharp}}) &=  \Ynt, \R := \R \cup \{\gltsred Y{\tsymc{\Endl}}{\bot}\}
%    \\
%    \word'(\Close) &= \Ynt, \R := \R \cup \{\gltsred Y\Closel{\bot}\}
    \\
    \word'(\tlambda{\alpha}\kind T) &= \Ynt, \R := \R \cup \{\gltsred Y {\lambda\alpha\colon\kind}{\wordb{\TT}}\}
    \\
    \word'(\tiota) &= \Ynt, \R := \R \cup \{\gltsred Y\iota\emptyword\} \quad
                     \text{where } \typec{\iota}\neq \Skip,\End_{\typec{\sharp}}
    \\
    \word'(\typec{\iota\ T_1\cdots T_m}) &= \Ynt, \R := \R \cup \{\gltsred Y{\iota_j}{\wordb{\typec{T_j}}}\} \quad \text{where }\typec{\iota} = \function{}{}, \typec{\choice\{\overline{l_i}\}}, \typec{\varrecs{\overline{l_i}}}
    \\
    \word'(\MSGn\TT) &= \Ynt, \R := \R \cup \{\gltsred Y{\tsymc{\sharp_1}}{\wordb{\TT}\bot}, \gltsred Y{\sharp_2}{\emptyword}\}
    \\
    \word'(\quant\kind\TT) &= \Ynt, \R := \R \cup \{\gltsred Y{\tsymc{\lquant\kind}}{\wordb{\TT}\bot}
%    , \gltsred Y{\lquant\kind_2}{\emptyword}
    \}
    \\
    \word'(\Semi\ \TT) &= \Ynt, \R := \R \cup \{\gltsred Y{;_1}{\wordb{\TT}}\}
    \\
    \word'(\semit \TT \UT) &= \word(\TT)\word(\UT)
    \\
    \word'(\tdual{(\talpha\ \typec{T_1}\typec{\ldots}\typec{T_m})}) &= \Ynt, \R := \R \cup \{\gltsred Y{\Duall_1}{\wordb{\talpha\ \typec{T_1}\typec{\ldots}\typec{T_m}}}, \gltsred Y{\Duall_2}\emptyword\}
  \end{align*}
  $\Ynt$ is a fresh non-terminal symbol in all cases,\\ 
  $\nsymc \epsilon$ is the empty word,\\
  $\nsymc \bot$ is a non-terminal symbol without productions.\\
  \caption{Function word($\TT$).}
  \label{fig:word}
\end{figure}


%%% Local Variables:
%%% mode: latex
%%% TeX-master: "42-CR"
%%% End:

A production rule in $\mathcal{R}$ is written as $\ltsred{\Xnt}{a}{\ntdelta}$: the left side of the arrow must be a non-terminal, $\Xnt\in\mathcal{N}$, while the right side must be a word. Note that $\ntdelta$ can be the empty word. Grammars in GNF are \emph{simple} when for every non-terminal and terminal symbol there is at most one production $\ltsred{\Xnt}{a}{\ntdelta}$ \cite{KorenjakH66}.

The function $\word({\typec{T}})$, described in Figure \ref{fig:word}, is responsible for translating types to words of non-terminal symbols. First consider a type in weak head normal form, that is, $\iswhnf{T}$. The construction of $\word(\TT)$ will create a fresh new terminal, $\Ynt$, and update the set of productions of $\TT$, accordingly to one of the cases found in $\word'$. In fact, we only had a new fresh non terminal if the type visited is syntactically different from all previously visited types. If $\TT$ normalises to $\Skip$, i.e. $\normalred T\Skip$, we return the empty word. Otherwise, if there exists a type $\UT$ different from $\Skip$ such that $T$ normalises to $U$, $\normalred TU \neq \Skip$, consider $\word(\UT) = \Znt \ntdelta$ and $\Ynt$ a fresh new terminal. For each production of $Z$ of the form $\gltsred Z a {\gamma}$, then $\Ynt$ has a production of the form $\gltsred Y a {\gamma\delta}$. 
This function terminates producing a simple grammar. \todo{Elaborate on why we can always produce a simple grammar - Reviewer 2}     
After this translation, we have to check if two grammars are bisimilar, that is if $\word({\typec{T}}) \gequiv \word({\typec{U}})$. The algorithm used to check the bisimilarity of simple grammars is the one introduced by Almeida et al. \cite{AlmeidaMV20}.

\todo{grammar example: T$\sim$U}
